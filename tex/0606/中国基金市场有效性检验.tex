%@Author:RenQingjie
%@Date:2017-06-1111:06:44
%@LastModifiedby:RenQingjie
%@LastModifiedtime:2017-06-1118:13:24

\documentclass[11pt]{article}
\usepackage[UTF8]{ctex}
\usepackage{apacite}
\usepackage{amsmath}
\usepackage{longtable,tabularx,multirow}
\usepackage{hyperref}
\hypersetup{colorlinks=true}
\usepackage{geometry}
\geometry{top=2.54cm,bottom=2.54cm,left=3cm,right=3cm}
\author{任庆杰 1400015500 renqingjie@pku.edu.cn}
\title{中国基金市场弱有效性检验}

\begin{document}

\maketitle
\section*{引言}
所谓市场有效率,是说在确定资产价格时能使用所获得的全部信息,或者说价格已经充分反映了所有可获得的全部信息。本文使用三支基金的日收盘数据,通过检验他们是否满足随机游走序列的假定,来判断基金市场是否为弱有效的。

\section{市场有效性理论及检验方法}
\subsection{市场有效性假说}
美国芝加哥大学教授法玛(Fama,1970)将有效市场分为三种形式:(1)弱型有效市场,其价格充分反映了市场的过去所有信息,包括历史价格走势和交易情况等;(2)半强型有效市场,包括宏观经济情况和相关政策等;(3)强行有效市场,其价格充分反映了市场的所有公开和未公开的信息,包括一切信息资料,是最大程度的市场效率。

如果一个市场是弱有效市场,则意味着其现在的市场价格变动与以前的价格变动没有关系。不可能用过去的价格变动来预测未来的价格走势,即此时技术分析是无效的。如果一个市场是半强型有效市场或者强行有效市场,那么不可能利用公开的信息获取超额利润,即此时基本面分析是无效的。

一个有效的市场是一个规范化的、竞争性的市场,它使得所有信息得到充分反映,价格随信息的变化而迅速调整,与价值相符,达到资本资源的有效配置和最优化。

法玛的理论成为研究如何规范完善证券市场的理论依据。许多国家根据他的理论,运用各种不同的方法,建立评价模型,进行实证研究,得到了大量的有用的结论,为证券市场的监督管理提供了实证依据。

\subsection{弱式有效性市场检验方法}
通常有两条途径对弱势有效市场进行检验:一是统计检验方法,即从统计学的角度找出“价格是否已经反映了所有可以得到的信息”的意义。这种方法通常给出一个随机模型,然后检验市场价格的波动是否符合这一给定的模型。而是技术分析方法,即从投资者的角度设计一种投资策略,然后按照这种策略买卖证券,最后与均衡条件下的证券收益相比较,检验是否能赚取超额利润。

\subsubsection{统计检验方法}
在统计检验中,通常使用随机游走模型,其含义是“无条件收益率与给定信息集$\Phi_{t-1}$条件下的收益率具有相同的概率分布。

以收益率表示的随机游走模型为:
\begin{equation}
f(R_{1,t},R_{2,t}\cdots R_{n,t})=f(R_{1,t},R_{2,t}\cdots R_{n,t}|\Phi_{t-1})
\end{equation}

以价格表示的随机游走模型为:
\begin{equation}
f(P_{1,t},P_{2,t}\cdots P_{n,t})=f(P_{1,t},P_{2,t}\cdots P_{n,t}|\Phi_{t-1})
\end{equation}

随机游走模型实际上包含了两个独立的假设:(1)连续的价格波动之间相互独立;(2)价格波动遵循相同的分布。并且要求两种概率分布的所有参数必须相等,包括均值、方差、偏斜度、峰态等。

\subsubsection{技术分析方法}
第二种检验若是有效市场的方法是技术分析法,其中具有代表性的有过滤法则、相对强度法。

1.\textbf{过滤法则}。这是最普通的一种技术分析方法,是Alexander在1961年首创的。其基本策略是:股价从某一谷底上涨超过某一预定幅度后卖出。这一过程一直持续下去,知道投资者破产或者赚到满意的收益为止。然后,将按过滤法则赚取的利润同按买入并持有策略(买入后一直持有,到最后卖出)赚取的利润进行比较,如果前者明显大于后者,则判断股价变动具有相关性,反之,则说明股价变动之间相互独立。Alexander的研究结果表明,过滤法则能给投资者带来巨额收益。

2.\textbf{相对强度法}。相对强度法是众多利用过去历史价格数据来选择股票中使用比较多一种市场有效性检验方法。

设$\overline{P}_{kt}$为第k只股票在t时刻的前T时期价格平均值;$P_{kt}$为第k只股票在t时刻的前T时期价格的实际值。

令$\alpha=\frac{\overline{P}_{kt}}{P_{kt}}$,则$\alpha$为相对强度。然后把所有可选择股票的各个$\alpha$计算出来,在选择$\alpha$值最大的前5\%,按等比例投资构成以证券资产组合。若某一股票的$\alpha$下降到最大$\alpha$值的前70\%以下,则卖出该股票,重新组织新的证券资产组合。

\subsection{随机游走的统计检验}
本文通过市场价格波动是否表现为随机游走来验证市场的弱有效性。如果市场价格变动服从随机游走过程,那么弱型有效市场假设是成立的,对市场价格变动服从随机游走过程检验的统计方法,采用序列相关检验和游程检验。

\subsubsection{序列相关检验}
考虑如下模型:
\begin{equation}
R_t=r_t+e_t
\end{equation}
其中,$R_t$表示收益率(从t-1期到t期价格变动而产生的收益率),$r_t$表示正常的收益率,$e_t$表示其随机误差项。

如果市场是弱有效的,即市场价格变动服从随机游走过程,则其收益率的误差项$e_t$不存在序列相关,也就是说下式成立:
\begin{equation}
cov(e_t,e_{t-k})=0,\quad k=1,2,\cdots ,t-1
\end{equation}
可以采用Ljung-Box\quad Q统计量:
\begin{equation}
Q=T(T+2)\sum_{k=1}^{m}\frac{\rho_k^2}{T-k}
\end{equation}
其中,$\rho_k^2$是误差滞后项i阶的样本自回归系数(i阶的自相关系数),T为样本个数。

此方法可以联合检验序列1-m阶是否存在序列相关。在误差项不存在序列相关的假设下,即(2)式成立的条件下,Q统计量服从自由度为m的$\chi^2$分布。因此,在取得数据,得到收益率的误差后,计算出Q统计量的值,以此可以判断误差项$e_t$不存在序列相关的假设是否成立,进而得出市场价格波动服从随机游走过程是否成立,市场是否有效。

\subsubsection{游程检验}
游程检验是另一种随机游走的检验方法,游程被定义为价格连续地单项运行过程,价格连续上升成为一个上升游程,价格连续下降称为一个下降游程。上升游程和下降游程个数之和称为总游程数。

总游程数S的均值$E(S)$和标准差$\sigma_S$分别为:
\begin{equation}
E(S)=\frac{N+2N_AN_B}{N}
\end{equation}
\begin{equation}
\sigma_S=\sqrt{\frac{2N_AN_B(2N_AN_B-N)}{N(N-1)}}
\end{equation}
其中:$N$为价格变动的总天数,$N_A$为价格上升的天数,$N_B$为价格下降的天数。

在市场价格变动服从随机游走过程的假设下,当$N$足够大时,$S$近似服从正态分布,因此可以定义Z统计量:
\begin{equation}
Z=\frac{S-E(S)}{\sigma_S}
\end{equation}
从而利用Z统计量来判断价格变动服从随机游走的假设是否成立。

\section{基金市场弱有效性的实证检验}

从基金市场中随机选取三支基金:长安鑫利(001281)、华夏大盘(000011)、工银瑞信国企(001008)。数据来自万得数据库,选取三支基金2016年1月4日至今的所有收盘价格。样本数据采用各基金每日收盘价($P_t$)。分别考虑其百分比($\dfrac{(P_t-P_{t-1}}{P_{t-1}})$)、对数差($\log P_t-\log P_{t-1}$)作为其价格波动的收益率。市场有效性检验的两种检验方法\textendash序列相关检验和游程检验的实证结果分别见表\ref{tb:acf}和表\ref{tb:run}。

从表\ref{tb:acf}可以看出,六种情形下的Q统计量均小于其相应的临界值,因此接受假设,可以得出三个基金品种的价格波动服从随机游走过程。

从表\ref{tb:run}中看出,所有的Z统计量的值均小于其相应的临界值1.96(p>0.05),所以接受原假设。即可以认为三个基金品种的价格波动服从随机游走过程。
%序列相关的检验
\begin{table}[h!]
\centering
\caption{自相关系数及检验结果}
\label{tb:acf}
\begin{tabular}{lllllll}\hline\hline
品种&序列&1阶自相关&5阶自相关&10阶自相关&Q统计量&p值\\\hline
长安鑫利&百分比&0.092&0.023&0.109&58.19&0.03\\
长安鑫利&对数差&0.091&0.022&0.108&57.34&0.04\\
华夏大盘&百分比&-0.179&-0.187&-0.074&119.31&0\\
华夏大盘&对数差&-0.179&-0.187&-0.072&121.1&0\\
工银瑞信国企&百分比&-0.121&-0.179&-0.062&93.96&0\\
工银瑞信国企&对数差&-0.121&-0.178&-0.060&95.33&0\\\hline\hline
\end{tabular}
\end{table}


%游程检验
\begin{table}[h!]
\centering
\caption{游程检验结果}
\label{tb:run}
\begin{tabular}{llll}\hline\hline
品种&游程数&Z统计量&p值\\\hline
长安鑫利&180&·1.01&0.31\\
华夏大盘&174&-0.11&0.91\\
工银瑞信国企&191&1.72&0.09\\
\hline\hline
\end{tabular}
\end{table}

\section{结论与展望}
本文采用了三支基金的日收盘价格数据,对基金市场的弱有效性进行了实证研究,序列相关检验和游程检验的方法说明,市场价格波动服从随机游走过程。因此,应该接受基金市场是弱型有效市场的假设。

但是,一方面,由于本研究使用的是日度数据,考虑到共同基金市场资本的流动性相对较强,技术套利的操作可能更为迅速。所以,仅仅日数据来说明基金市场中技术分析无效可能缺乏说服力。

另外,由于时间和精力的原因,本文只选取了3支基金进行时间序列的分析。这样得到的结果,可能是不够稳健的。在后面的研究中,可以考虑引入截面模型和面板数据,以得到更加可靠的结论
\clearpage
\appendix
% @Author: Ren Qingjie
% @Date:   2017-06-11 17:55:45
% @Last Modified by:   Ren Qingjie
% @Last Modified time: 2017-06-11 18:11:26

\section{数据}
\begin{center}
\begin{longtable}{r lll}
    \caption{三只基金日收盘价数据}
    \label{tb:rawdata} \\
     \hline \hline
    date       & 001281 & 000011 & 001008 \\ \hline
    2016-01-04 & 1.08   & 10.62  & 0.97   \\
    2016-01-05 & 1.079  & 10.544 & 0.98   \\
    2016-01-06 & 1.081  & 10.851 & 1.003  \\
    2016-01-07 & 1.076  & 9.966  & 0.926  \\
    2016-01-08 & 1.076  & 10.159 & 0.951  \\
    2016-01-11 & 1.072  & 9.623  & 0.893  \\
    2016-01-12 & 1.072  & 9.661  & 0.884  \\
    2016-01-13 & 1.07   & 9.37   & 0.857  \\
    2016-01-14 & 1.072  & 9.67   & 0.874  \\
    2016-01-15 & 1.07   & 9.233  & 0.84   \\
    2016-01-18 & 1.071  & 9.277  & 0.845  \\
    2016-01-19 & 1.074  & 9.63   & 0.878  \\
    2016-01-20 & 1.073  & 9.454  & 0.864  \\
    2016-01-21 & 1.07   & 9.081  & 0.833  \\
    2016-01-22 & 1.072  & 9.254  & 0.843  \\
    2016-01-25 & 1.072  & 9.449  & 0.855  \\
    2016-01-26 & 1.062  & 8.844  & 0.805  \\
    2016-01-27 & 1.058  & 8.876  & 0.798  \\
    2016-01-28 & 1.051  & 8.526  & 0.773  \\
    2016-01-29 & 1.057  & 8.79   & 0.798  \\
    2016-02-01 & 1.055  & 8.664  & 0.791  \\
    2016-02-02 & 1.061  & 8.911  & 0.811  \\
    2016-02-03 & 1.061  & 8.888  & 0.812  \\
    2016-02-04 & 1.064  & 9.006  & 0.825  \\
    2016-02-05 & 1.061  & 8.962  & 0.818  \\
    2016-02-15 & 1.062  & 8.961  & 0.813  \\
    2016-02-16 & 1.068  & 9.27   & 0.84   \\
    2016-02-17 & 1.07   & 9.325  & 0.847  \\
    2016-02-18 & 1.069  & 9.352  & 0.844  \\
    2016-02-19 & 1.071  & 9.406  & 0.847  \\
    2016-02-22 & 1.074  & 9.672  & 0.868  \\
    2016-02-23 & 1.072  & 9.76   & 0.857  \\
    2016-02-24 & 1.072  & 9.778  & 0.866  \\
    2016-02-25 & 1.062  & 9.06   & 0.794  \\
    2016-02-26 & 1.062  & 9.286  & 0.798  \\
    2016-02-29 & 1.055  & 8.991  & 0.757  \\
    2016-03-01 & 1.057  & 9.16   & 0.771  \\
    2016-03-02 & 1.063  & 9.629  & 0.807  \\
    2016-03-03 & 1.064  & 9.573  & 0.808  \\
    2016-03-04 & 1.06   & 9.417  & 0.793  \\
    2016-03-07 & 1.062  & 9.58   & 0.807  \\
    2016-03-08 & 1.064  & 9.518  & 0.81   \\
    2016-03-09 & 1.062  & 9.231  & 0.792  \\
    2016-03-10 & 1.059  & 9.117  & 0.781  \\
    2016-03-11 & 1.059  & 9.125  & 0.779  \\
    2016-03-14 & 1.064  & 9.359  & 0.805  \\
    2016-03-15 & 1.064  & 9.262  & 0.801  \\
    2016-03-16 & 1.062  & 9.182  & 0.794  \\
    2016-03-17 & 1.066  & 9.437  & 0.816  \\
    2016-03-18 & 1.07   & 9.684  & 0.838  \\
    2016-03-21 & 1.073  & 9.852  & 0.856  \\
    2016-03-22 & 1.072  & 9.8    & 0.85   \\
    2016-03-23 & 1.074  & 9.833  & 0.86   \\
    2016-03-24 & 1.072  & 9.728  & 0.848  \\
    2016-03-25 & 1.073  & 9.756  & 0.852  \\
    2016-03-28 & 1.072  & 9.755  & 0.85   \\
    2016-03-29 & 1.07   & 9.657  & 0.835  \\
    2016-03-30 & 1.075  & 9.937  & 0.859  \\
    2016-03-31 & 1.074  & 9.961  & 0.864  \\
    2016-04-01 & 1.073  & 9.925  & 0.858  \\
    2016-04-05 & 1.076  & 10.115 & 0.876  \\
    2016-04-06 & 1.076  & 10.144 & 0.879  \\
    2016-04-07 & 1.074  & 10.033 & 0.864  \\
    2016-04-08 & 1.073  & 9.937  & 0.857  \\
    2016-04-11 & 1.075  & 10.144 & 0.876  \\
    2016-04-12 & 1.075  & 10.114 & 0.875  \\
    2016-04-13 & 1.076  & 10.274 & 0.888  \\
    2016-04-14 & 1.077  & 10.32  & 0.895  \\
    2016-04-15 & 1.077  & 10.288 & 0.893  \\
    2016-04-18 & 1.076  & 10.149 & 0.876  \\
    2016-04-19 & 1.076  & 10.164 & 0.878  \\
    2016-04-20 & 1.074  & 9.838  & 0.847  \\
    2016-04-21 & 1.073  & 9.854  & 0.838  \\
    2016-04-22 & 1.073  & 9.858  & 0.84   \\
    2016-04-25 & 1.073  & 9.9    & 0.835  \\
    2016-04-26 & 1.073  & 9.959  & 0.844  \\
    2016-04-27 & 1.074  & 9.904  & 0.84   \\
    2016-04-28 & 1.074  & 9.871  & 0.837  \\
    2016-04-29 & 1.073  & 9.957  & 0.838  \\
    2016-05-03 & 1.075  & 10.254 & 0.86   \\
    2016-05-04 & 1.076  & 10.189 & 0.857  \\
    2016-05-05 & 1.076  & 10.199 & 0.859  \\
    2016-05-06 & 1.073  & 9.85   & 0.829  \\
    2016-05-09 & 1.071  & 9.504  & 0.803  \\
    2016-05-10 & 1.071  & 9.488  & 0.802  \\
    2016-05-11 & 1.07   & 9.55   & 0.802  \\
    2016-05-12 & 1.071  & 9.581  & 0.804  \\
    2016-05-13 & 1.07   & 9.542  & 0.8    \\
    2016-05-16 & 1.072  & 9.703  & 0.809  \\
    2016-05-17 & 1.072  & 9.642  & 0.804  \\
    2016-05-18 & 1.07   & 9.416  & 0.787  \\
    2016-05-19 & 1.071  & 9.449  & 0.785  \\
    2016-05-20 & 1.072  & 9.513  & 0.793  \\
    2016-05-23 & 1.072  & 9.617  & 0.804  \\
    2016-05-24 & 1.072  & 9.541  & 0.794  \\
    2016-05-25 & 1.072  & 9.462  & 0.792  \\
    2016-05-26 & 1.072  & 9.504  & 0.796  \\
    2016-05-27 & 1.072  & 9.523  & 0.796  \\
    2016-05-30 & 1.072  & 9.43   & 0.794  \\
    2016-05-31 & 1.076  & 9.737  & 0.823  \\
    2016-06-01 & 1.077  & 9.792  & 0.824  \\
    2016-06-02 & 1.077  & 9.845  & 0.829  \\
    2016-06-03 & 1.077  & 9.891  & 0.834  \\
    2016-06-06 & 1.078  & 9.923  & 0.833  \\
    2016-06-07 & 1.078  & 9.922  & 0.837  \\
    2016-06-08 & 1.078  & 9.899  & 0.832  \\
    2016-06-13 & 1.074  & 9.615  & 0.8    \\
    2016-06-14 & 1.075  & 9.632  & 0.805  \\
    2016-06-15 & 1.077  & 9.878  & 0.823  \\
    2016-06-16 & 1.078  & 9.881  & 0.818  \\
    2016-06-17 & 1.078  & 9.929  & 0.828  \\
    2016-06-20 & 1.079  & 9.924  & 0.827  \\
    2016-06-21 & 1.078  & 9.832  & 0.826  \\
    2016-06-22 & 1.079  & 9.966  & 0.838  \\
    2016-06-23 & 1.078  & 9.876  & 0.834  \\
    2016-06-24 & 1.078  & 9.818  & 0.825  \\
    2016-06-27 & 1.08   & 9.998  & 0.841  \\
    2016-06-28 & 1.081  & 10.042 & 0.851  \\
    2016-06-29 & 1.08   & 10.07  & 0.852  \\
    2016-06-30 & 1.081  & 10.033 & 0.853  \\
    2016-07-01 & 1.081  & 10.068 & 0.851  \\
    2016-07-04 & 1.083  & 10.25  & 0.867  \\
    2016-07-05 & 1.083  & 10.295 & 0.871  \\
    2016-07-06 & 1.083  & 10.426 & 0.876  \\
    2016-07-07 & 1.083  & 10.476 & 0.877  \\
    2016-07-08 & 1.083  & 10.405 & 0.874  \\
    2016-07-11 & 1.083  & 10.451 & 0.875  \\
    2016-07-12 & 1.084  & 10.621 & 0.888  \\
    2016-07-13 & 1.085  & 10.64  & 0.893  \\
    2016-07-14 & 1.085  & 10.594 & 0.89   \\
    2016-07-15 & 1.084  & 10.599 & 0.888  \\
    2016-07-18 & 1.084  & 10.555 & 0.882  \\
    2016-07-19 & 1.083  & 10.512 & 0.884  \\
    2016-07-20 & 1.084  & 10.467 & 0.88   \\
    2016-07-21 & 1.084  & 10.465 & 0.883  \\
    2016-07-22 & 1.084  & 10.36  & 0.873  \\
    2016-07-25 & 1.084  & 10.349 & 0.873  \\
    2016-07-26 & 1.085  & 10.544 & 0.889  \\
    2016-07-27 & 1.081  & 10.251 & 0.863  \\
    2016-07-28 & 1.08   & 10.332 & 0.864  \\
    2016-07-29 & 1.08   & 10.293 & 0.862  \\
    2016-08-01 & 1.078  & 10.214 & 0.853  \\
    2016-08-02 & 1.079  & 10.265 & 0.862  \\
    2016-08-03 & 1.079  & 10.324 & 0.865  \\
    2016-08-04 & 1.08   & 10.31  & 0.865  \\
    2016-08-05 & 1.079  & 10.286 & 0.863  \\
    2016-08-08 & 1.081  & 10.297 & 0.875  \\
    2016-08-09 & 1.082  & 10.373 & 0.88   \\
    2016-08-10 & 1.081  & 10.396 & 0.879  \\
    2016-08-11 & 1.079  & 10.287 & 0.876  \\
    2016-08-12 & 1.08   & 10.339 & 0.886  \\
    2016-08-15 & 1.083  & 10.478 & 0.91   \\
    2016-08-16 & 1.083  & 10.515 & 0.91   \\
    2016-08-17 & 1.084  & 10.518 & 0.91   \\
    2016-08-18 & 1.083  & 10.514 & 0.903  \\
    2016-08-19 & 1.083  & 10.52  & 0.903  \\
    2016-08-22 & 1.082  & 10.461 & 0.891  \\
    2016-08-23 & 1.082  & 10.481 & 0.893  \\
    2016-08-24 & 1.082  & 10.476 & 0.894  \\
    2016-08-25 & 1.081  & 10.421 & 0.889  \\
    2016-08-26 & 1.081  & 10.443 & 0.893  \\
    2016-08-29 & 1.082  & 10.486 & 0.892  \\
    2016-08-30 & 1.082  & 10.484 & 0.891  \\
    2016-08-31 & 1.082  & 10.478 & 0.89   \\
    2016-09-01 & 1.081  & 10.498 & 0.878  \\
    2016-09-02 & 1.081  & 10.465 & 0.876  \\
    2016-09-05 & 1.081  & 10.495 & 0.881  \\
    2016-09-06 & 1.083  & 10.585 & 0.895  \\
    2016-09-07 & 1.083  & 10.532 & 0.896  \\
    2016-09-08 & 1.084  & 10.505 & 0.899  \\
    2016-09-09 & 1.083  & 10.489 & 0.893  \\
    2016-09-12 & 1.08   & 10.242 & 0.869  \\
    2016-09-13 & 1.081  & 10.261 & 0.867  \\
    2016-09-14 & 1.08   & 10.229 & 0.863  \\
    2016-09-19 & 1.082  & 10.265 & 0.875  \\
    2016-09-20 & 1.081  & 10.288 & 0.876  \\
    2016-09-21 & 1.082  & 10.289 & 0.882  \\
    2016-09-22 & 1.082  & 10.337 & 0.888  \\
    2016-09-23 & 1.082  & 10.305 & 0.885  \\
    2016-09-26 & 1.079  & 10.138 & 0.87   \\
    2016-09-27 & 1.08   & 10.153 & 0.874  \\
    2016-09-28 & 1.08   & 10.168 & 0.872  \\
    2016-09-29 & 1.081  & 10.203 & 0.877  \\
    2016-09-30 & 1.081  & 10.221 & 0.883  \\
    2016-10-10 & 1.084  & 10.42  & 0.893  \\
    2016-10-11 & 1.085  & 10.556 & 0.898  \\
    2016-10-12 & 1.084  & 10.521 & 0.895  \\
    2016-10-13 & 1.085  & 10.522 & 0.893  \\
    2016-10-14 & 1.084  & 10.504 & 0.894  \\
    2016-10-17 & 1.083  & 10.456 & 0.886  \\
    2016-10-18 & 1.085  & 10.549 & 0.899  \\
    2016-10-19 & 1.084  & 10.515 & 0.898  \\
    2016-10-20 & 1.084  & 10.514 & 0.898  \\
    2016-10-21 & 1.083  & 10.458 & 0.895  \\
    2016-10-24 & 1.084  & 10.521 & 0.909  \\
    2016-10-25 & 1.084  & 10.601 & 0.917  \\
    2016-10-26 & 1.084  & 10.596 & 0.915  \\
    2016-10-27 & 1.084  & 10.591 & 0.912  \\
    2016-10-28 & 1.083  & 10.539 & 0.907  \\
    2016-10-31 & 1.083  & 10.562 & 0.908  \\
    2016-11-01 & 1.084  & 10.728 & 0.914  \\
    2016-11-02 & 1.083  & 10.674 & 0.906  \\
    2016-11-03 & 1.084  & 10.667 & 0.916  \\
    2016-11-04 & 1.083  & 10.633 & 0.912  \\
    2016-11-07 & 1.083  & 10.664 & 0.917  \\
    2016-11-08 & 1.084  & 10.678 & 0.916  \\
    2016-11-09 & 1.083  & 10.667 & 0.91   \\
    2016-11-10 & 1.084  & 10.734 & 0.921  \\
    2016-11-11 & 1.085  & 10.765 & 0.934  \\
    2016-11-14 & 1.085  & 10.787 & 0.943  \\
    2016-11-15 & 1.086  & 10.798 & 0.938  \\
    2016-11-16 & 1.086  & 10.841 & 0.939  \\
    2016-11-17 & 1.086  & 10.842 & 0.941  \\
    2016-11-18 & 1.086  & 10.8   & 0.933  \\
    2016-11-21 & 1.085  & 10.753 & 0.936  \\
    2016-11-22 & 1.086  & 10.842 & 0.946  \\
    2016-11-23 & 1.086  & 10.783 & 0.939  \\
    2016-11-24 & 1.085  & 10.758 & 0.941  \\
    2016-11-25 & 1.085  & 10.767 & 0.944  \\
    2016-11-28 & 1.085  & 10.801 & 0.948  \\
    2016-11-29 & 1.085  & 10.721 & 0.946  \\
    2016-11-30 & 1.084  & 10.658 & 0.935  \\
    2016-12-01 & 1.085  & 10.719 & 0.943  \\
    2016-12-02 & 1.083  & 10.605 & 0.925  \\
    2016-12-05 & 1.082  & 10.571 & 0.915  \\
    2016-12-06 & 1.082  & 10.607 & 0.914  \\
    2016-12-07 & 1.084  & 10.713 & 0.922  \\
    2016-12-08 & 1.082  & 10.691 & 0.919  \\
    2016-12-09 & 1.082  & 10.65  & 0.922  \\
    2016-12-12 & 1.075  & 10.294 & 0.891  \\
    2016-12-13 & 1.076  & 10.194 & 0.896  \\
    2016-12-14 & 1.075  & 10.138 & 0.89   \\
    2016-12-15 & 1.075  & 10.118 & 0.885  \\
    2016-12-16 & 1.076  & 10.14  & 0.888  \\
    2016-12-19 & 1.076  & 10.124 & 0.886  \\
    2016-12-20 & 1.075  & 10.082 & 0.88   \\
    2016-12-21 & 1.076  & 10.175 & 0.888  \\
    2016-12-22 & 1.077  & 10.197 & 0.889  \\
    2016-12-23 & 1.076  & 10.112 & 0.878  \\
    2016-12-26 & 1.076  & 10.13  & 0.884  \\
    2016-12-27 & 1.076  & 10.147 & 0.883  \\
    2016-12-28 & 1.076  & 10.136 & 0.881  \\
    2016-12-29 & 1.076  & 10.097 & 0.879  \\
    2016-12-30 & 1.076  & 10.131 & 0.881  \\
    2016-12-31 & 1.076  & 10.13  & 0.881  \\
    2017-01-03 & 1.077  & 10.222 & 0.889  \\
    2017-01-04 & 1.084  & 10.319 & 0.896  \\
    2017-01-05 & 1.082  & 10.331 & 0.899  \\
    2017-01-06 & 1.08   & 10.262 & 0.895  \\
    2017-01-09 & 1.082  & 10.304 & 0.901  \\
    2017-01-10 & 1.082  & 10.309 & 0.899  \\
    2017-01-11 & 1.079  & 10.189 & 0.892  \\
    2017-01-12 & 1.08   & 10.09  & 0.884  \\
    2017-01-13 & 1.086  & 9.954  & 0.881  \\
    2017-01-16 & 1.097  & 9.774  & 0.874  \\
    2017-01-17 & 1.097  & 9.84   & 0.876  \\
    2017-01-18 & 1.103  & 9.872  & 0.879  \\
    2017-01-19 & 1.103  & 9.829  & 0.879  \\
    2017-01-20 & 1.105  & 9.971  & 0.885  \\
    2017-01-23 & 1.102  & 10.036 & 0.887  \\
    2017-01-24 & 1.109  & 9.987  & 0.886  \\
    2017-01-25 & 1.11   & 10.044 & 0.888  \\
    2017-01-26 & 1.114  & 10.101 & 0.893  \\
    2017-02-03 & 1.105  & 10.129 & 0.888  \\
    2017-02-06 & 1.104  & 10.187 & 0.892  \\
    2017-02-07 & 1.101  & 10.193 & 0.892  \\
    2017-02-08 & 1.102  & 10.222 & 0.898  \\
    2017-02-09 & 1.105  & 10.271 & 0.899  \\
    2017-02-10 & 1.11   & 10.244 & 0.904  \\
    2017-02-13 & 1.115  & 10.283 & 0.913  \\
    2017-02-14 & 1.115  & 10.277 & 0.916  \\
    2017-02-15 & 1.125  & 10.189 & 0.911  \\
    2017-02-16 & 1.126  & 10.251 & 0.915  \\
    2017-02-17 & 1.118  & 10.176 & 0.905  \\
    2017-02-20 & 1.135  & 10.336 & 0.92   \\
    2017-02-21 & 1.136  & 10.379 & 0.924  \\
    2017-02-22 & 1.14   & 10.436 & 0.928  \\
    2017-02-23 & 1.14   & 10.401 & 0.922  \\
    2017-02-24 & 1.138  & 10.393 & 0.921  \\
    2017-02-27 & 1.132  & 10.339 & 0.915  \\
    2017-02-28 & 1.135  & 10.414 & 0.921  \\
    2017-03-01 & 1.136  & 10.467 & 0.925  \\
    2017-03-02 & 1.131  & 10.555 & 0.92   \\
    2017-03-03 & 1.132  & 10.568 & 0.922  \\
    2017-03-06 & 1.145  & 10.652 & 0.924  \\
    2017-03-07 & 1.148  & 10.728 & 0.924  \\
    2017-03-08 & 1.144  & 10.711 & 0.926  \\
    2017-03-09 & 1.137  & 10.613 & 0.917  \\
    2017-03-10 & 1.139  & 10.685 & 0.916  \\
    2017-03-13 & 1.149  & 10.753 & 0.923  \\
    2017-03-14 & 1.149  & 10.838 & 0.923  \\
    2017-03-15 & 1.1518 & 10.857 & 0.922  \\
    2017-03-16 & 1.1564 & 10.93  & 0.929  \\
    2017-03-17 & 1.1436 & 10.849 & 0.92   \\
    2017-03-20 & 1.1525 & 10.88  & 0.921  \\
    2017-03-21 & 1.1713 & 11.011 & 0.929  \\
    2017-03-22 & 1.1673 & 11.028 & 0.927  \\
    2017-03-23 & 1.1729 & 11.073 & 0.926  \\
    2017-03-24 & 1.185  & 11.126 & 0.931  \\
    2017-03-27 & 1.1838 & 11.005 & 0.925  \\
    2017-03-28 & 1.1805 & 10.943 & 0.921  \\
    2017-03-29 & 1.1812 & 10.976 & 0.922  \\
    2017-03-30 & 1.1787 & 10.899 & 0.913  \\
    2017-03-31 & 1.1882 & 10.937 & 0.92   \\
    2017-04-05 & 1.1979 & 11.05  & 0.939  \\
    2017-04-06 & 1.2024 & 11.074 & 0.95   \\
    2017-04-07 & 1.2047 & 11.133 & 0.954  \\
    2017-04-10 & 1.2019 & 11.047 & 0.961  \\
    2017-04-11 & 1.1977 & 11.019 & 0.967  \\
    2017-04-12 & 1.2034 & 10.969 & 0.959  \\
    2017-04-13 & 1.2074 & 11.044 & 0.967  \\
    2017-04-14 & 1.1998 & 10.943 & 0.959  \\
    2017-04-17 & 1.1943 & 10.894 & 0.957  \\
    2017-04-18 & 1.201  & 10.981 & 0.958  \\
    2017-04-19 & 1.1993 & 10.979 & 0.952  \\
    2017-04-20 & 1.2182 & 11.125 & 0.957  \\
    2017-04-21 & 1.2115 & 11.041 & 0.955  \\
    2017-04-24 & 1.207  & 10.921 & 0.94   \\
    2017-04-25 & 1.2177 & 11.022 & 0.947  \\
    2017-04-26 & 1.2188 & 10.97  & 0.947  \\
    2017-04-27 & 1.2183 & 10.958 & 0.951  \\
    2017-04-28 & 1.2072 & 10.912 & 0.948  \\
    2017-05-02 & 1.2065 & 10.967 & 0.943  \\
    2017-05-03 & 1.2026 & 10.963 & 0.937  \\
    2017-05-04 & 1.2032 & 10.974 & 0.932  \\
    2017-05-05 & 1.1999 & 10.959 & 0.922  \\
    2017-05-08 & 1.1891 & 10.837 & 0.912  \\
    2017-05-09 & 1.193  & 10.81  & 0.914  \\
    2017-05-10 & 1.2016 & 10.793 & 0.907  \\
    2017-05-11 & 1.213  & 10.908 & 0.913  \\
    2017-05-12 & 1.2288 & 10.949 & 0.918  \\
    2017-05-15 & 1.242  & 10.983 & 0.924  \\
    2017-05-16 & 1.2543 & 11.196 & 0.934  \\
    2017-05-17 & 1.2373 & 11.138 & 0.931  \\
    2017-05-18 & 1.2378 & 11.094 & 0.927  \\
    2017-05-19 & 1.2475 & 11.124 & 0.928  \\
    2017-05-22 & 1.2542 & 11.174 & 0.924  \\
    2017-05-23 & 1.267  & 11.262 & 0.917  \\
    2017-05-24 & 1.2623 & 11.21  & 0.916  \\
    2017-05-25 & 1.2718 & 11.181 & 0.932  \\
    2017-05-26 & 1.2753 & 11.174 & 0.934  \\
    2017-05-31 & 1.273  & 11.181 & 0.935  \\
    2017-06-01 & 1.2884 & 11.232 & 0.934  \\
    2017-06-02 & 1.2827 & 11.184 & 0.932  \\
    2017-06-05 & 1.2712 & 11.198 & 0.929  \\
    2017-06-06 & 1.2887 & 11.338 & 0.934  \\
    2017-06-07 & 1.3156 & 11.503 & 0.949  \\
    2017-06-08 & 1.3253 & 11.659 & 0.956  \\
    2017-06-09 & 1.341  & 11.712 & 0.959  \\
    \hline \hline
% \end{tabular}
\end{longtable}
\end{center}

\end{document}
