% @Author: Ren Qingjie
% @Date:   2017-05-31 19:22:11
% @Last Modified by:   Ren Qingjie
% @Last Modified time: 2017-06-15 10:11:20

\documentclass[10pt]{article}
    \usepackage{apacite} %apa格式引用
    \usepackage[section]{placeins} %防止图片浮动过Section
    \usepackage[UTF8]{ctex} %中文
    \usepackage{threeparttable} %多功能表格
    \usepackage{geometry} %边框
    \usepackage{appendix} %附录
    \usepackage{listings,xcolor}
    \usepackage{multicol}
    \usepackage{multirow}
    \usepackage{apacite} %apa格式引用
    \usepackage{framed} %插入代码和其它结果
    \lstset{frame=shadowbox,rulesepcolor=\color{red!20!green!20!blue!20}, %边框阴影
            numbers=left, %设置行号位置
            numberstyle=\tiny, %设置行号大小
            basicstyle=\tiny, %代码字体大小设定
            keywordstyle=\color{blue}, %设置关键字颜色
            commentstyle=\color[cmyk]{1,0,1,0}, %设置注释颜色
            %frame=single, %设置边框格式
            escapeinside=``, %逃逸字符(1左面的键),用于显示中文
            breaklines, %自动折行
            extendedchars=false, %解决代码跨页时,章节标题,页眉等汉字不显示的问题
            xleftmargin=2em,xrightmargin=2em, aboveskip=1em, %设置边距
            tabsize=4, %设置tab空格数
            showspaces=false %不显示空格
           }
    % \def\verbatim{\scriptsize\@verbatim \frenchspacing\@vobeyspaces \@xverbatim}
    \geometry{top=2.54cm,bottom=2.54cm,left=3cm,right=3cm}
    \author{任庆杰  140001500  renqingjie@pku.edu.cn}
    \title{经济全球化与地方政府规模\\
    ——基于中国地方经济数据的实证检验}

\begin{document}
\maketitle{}
\begin{framed}
    \begin{verbatim}
Average correlation coefficients & Pesaran (2004) CD test

Variables series tested: govsize
                           Group variable: pref
                         Number of groups: 318
                Average # of observations: 7.99
                                 Panel is: unbalanced

---------------------------------------------------------
Variable |    CD-test  p-value     corr  abs(corr)
-------------+-------------------------------------------
 govsize |     460.50    0.000    0.727    0.758
---------------------------------------------------------
Notes: Under the null hypothesis of cross-section 
    independence CD ~ N(0,1)
\end{verbatim}
\end{framed}
\end{document}
