% @Author: Ren Qingjie
% @Date:   2017-05-31 19:22:11
% @Last Modified by:   Ren Qingjie
% @Last Modified time: 2017-06-08 23:56:16

\documentclass[10pt]{article}
\usepackage{apacite}
\usepackage[UTF8]{ctex}
\usepackage{changes}
% \usepackage{threeparttable}
\usepackage{threeparttable}
\usepackage{geometry}
    \definechangesauthor[name={Jack}, color=orange]{per}
    \setremarkmarkup{(#2)}
    \geometry{top=3cm, bottom=3cm, left=3cm, right=3cm}
\author{任庆杰  140001500  renqingjie@pku.edu.cn}
\title{经济全球化与地方政府规模\\
——基于中国地方经济数据的实证检验}

\begin{document}
\maketitle{}
\section{引言}
在过去的30年里,人们越来越关注全球化现象。世界贸易组织、G20等全球性的经济机构和欧盟、亚太经合组织等地区性的经济合作组织在经济活动中的影响力越来越大。但人们对全球化的态度褒贬不一。特别是2008年全球性的金融危机爆发之后,世界的经济形势持续低迷,许多人将之归咎于全球化的冲击。而近年英国脱欧、特朗普的上台、以及许多国家爆发的民粹主义运动,全球化进程面临着严重的挑战,在不久的将来甚至可能戛然而止或者倒退。

关于“全球化”,并没有一个统一的定义。韦氏词典对“globalization”一词的解释为“the act or process of globalizing; the state of being globalized; especially: the development of an increasingly integrated global economy marked especially by free trade, free flow of capital, and the tapping of cheaper foreign labor markets”,这个定义特别强调了3点:自由贸易、资本自由流动和劳动力市场的选择。这三个方面侧重地反映了全球化对生产者的影响:可以更加自由地选择生产要素,可以低成本而大规模地生产。

本文则主要讨论全球化对地方政府规模和政府支出的影响。全球化可能增大政府和社会面临的风险,进而促使政府调整职能,来保护居民的权益,即可能的一种假设是,全球化改变了政府的职能,从而迫使政府增加政府支出。同时,考虑到全球化之后,企业可能更加自由地选择生产地和避税港,政府为了更好地吸引资本流入,会有减税和增加市场自由度的激励,即另一种假设是:全球化也可能会带来政府间的竞争,为了在这场竞争中取得优势,政府会相应地减小自身规模。本文使用中国区域统计年鉴和CEIC的数据,从实证方面对此进行检验。

\section{文献综述}
\subsection{政府职能的发展历史}
在过去的一个多世纪的时间里,受到全球经济形势和主流经济思潮变化的影响,政府职能在不断地发生变化,政府支出和政府规模也随之而发生了几次重大的转变。在第一次世界大战之前,政府承担的社会职能有限,许多社会保障功能由教会和社区承担,政府支出的比例也相对较小\cite{Beito2002The}。20年代的经济大萧条开始,凯恩斯主义成为指导经济和财政政策真理,政府支出开始不断增加,以增加社会总消费和总投资。尽管一些人对凯恩斯思想的范例“罗斯福新政”的效用持有怀疑态度\cite{rothbard1972america},但不可否认,在接下来的近半个世纪的时间内,几乎所有的国家都在走着凯恩斯设计的这条道路,政府支出开始不断增加。
直到70年代的滞胀,以哈耶克为代表的奥地利学派和以弗里德曼为代表的新古典经济学派开始逐渐在经济学向左向右论战中占据主动。而几乎同时期在大选中获胜的里根和撒切尔,则是自由主义学派的代表性的政治人物。而之后的一段时期内,中国经历了改革开放,从计划经济逐渐走向市场经济时代;苏联解体,曾经的国有企业被私有化;两个最大的计划经济体的转变,意味着没有任何一个经济体能够避开全球化的浪潮。政府将更多的权力交给了市场,政府支出和政府规模在这一段时间都开始呈现下降趋势。\added[remark={(表4.5)}]{}

直到2008年全球性的金融危机爆发,许多人开始呼吁政府更多地进入市场。一方面,全球性增大了金融危机的破坏力和传染力,许多人将此次危机归因于全球化进程,人们要求政府增大对金融机构的监管和对风险的抵抗能力;另一方面,当金融危机引发经济萧条,公司破产,失业率增加时,人们希望政府采取措施来恢复经济发展:包括减税以增加企业竞争力,也包括通过贸易保护措施来发展本土产业。此外,人们对政府职能提出了更新更高的要求,除了传统的做好市场经济的“守夜人”之外,人们也希望政府在社会保障如医疗、养老、教育等方面发挥更大的作用,促进社会平等和人权发展。
\subsection{全球化和政府规模的两种假说}
关于全球化和政府支出的关系,目前的研究存在两个相互矛盾的假说\cite{garrett2001globalization}。一个是效率假说,较高的政府支出,会降低其在全球化中的竞争力,因此随着全球化进程的推进,该经济体面临着经济的效率压力。此时政府会被迫采取更加开放的政策、降低税率、减少政府支出\cite{wangyu2014}。而除了效率的考虑之外,税基侵蚀也可能是政府规模减小的一个原因。全球化使得企业更加容易采取避税措施,在税基受到侵蚀的情况下,政府很难继续获得高税收收入\cite{tanzi2000globalization}。另一个补偿假说是说,随着全球化的加深,政府必须扩大支出来增加本地的竞争力,包括在教育上的投入以提高人力资本、增加公共品的供给以吸引资本流入\cite{rodrik1998more}。Rodrik还认为,对外贸易存在着更大的波动性,政府需要增大支出来应对潜在的风险,“对外依存度高的经济体,其政府支出扩张也较快” \cite{rodrik1998more}。但他忽略了对外贸易频繁,对外依存度高的经济体,往往会有更大的关税收入\cite{tanzi1973theory}。
\subsection{已有的实证检验}
不同的财政政策可能基于不同的路径,也可能同时兼顾而有所侧重。\cite{kettl2000transformation} 认为无论是从人民的期待方面,还是在实际的功能上,美国政府已经在全球化的进程中发生了改变;\cite{garrett2001globalization}使用18个OECD国家1961到1993年的面板数据,发现政府支出减少是贸易总量增加和金融开放度增加的重要原因,但是全球市场的流动和政府对资本与劳动的课税并没有显著的关系。\cite{dreher2008impact} 使用60个样本国家的1971到2001年的面板数据,进行验证,并没有发现全球化对政府支出有显著的影响。\cite{bergh2010government}使用了1970-1995和1970到2005两个时间段的面板数据,得到政府规模和经济增长负相关的结论;但同时也发现了大政府往往会通过使用经济开放政策来减小政府规模对经济增长的影响。\cite{kaufman2001globalization}通过使用拉丁美洲1973-1997年的面板数据,发现贸易总量和对资本的开放程度相关,而对政府在社会保障方面的支出有负向影响。\cite{maojie2015}利用1850—2009年的长时段跨国数据得到结论,政府提供社会安全网的职能越突出,经济开放对政府规模的正向影响越显著;反之,负向影响越显著。


此外,一些学者也对中国地方的开放程度和政府规模进行了研究。\cite{cai2008}利用中国27个省级行政单位1995-2004年的面板数据,检验全球化与地方政府支出结构的关系,得到结果外商直接投资限制了政府规模,贸易开放度扩大了政府规模。\cite{wang2007}研究得出,财政分权导致中国各个地方政府围绕经济指标而进行竞争,包括对外商投资的竞争。而由于中国东西部发展不平衡,西部地区的竞争方式主要措施为税收优惠和增加财政支出。并进一步探讨了这种竞争模式下外商投资对经济增长的促进作用。\cite{sheng2010}通过1978年到2008年不同省份地方变量的面板数据,分析了经济全球化造成的财政支出规模、结构和时段的非平衡性。
\subsection{本文的框架}
本文使用更加详细的来自地区级别的面板数据,来验证中国加入世界贸易组织以来,中国的地方政府规模和地方政府支出受到的全球化的影响。同时,考虑到2008年的特殊性,本文在2008年设置断点。2008年,世界性的金融危机爆发,地方政府为了刺激经济而采取了相应的财政政策,这是应当考虑的一个重要冲击变量;另外,在2008年,中国的企业所得税制度发生了重要的改变,中资和外资企业税率的统一,对政府税收收入产生了重要的影响。

\section{实证分析:模型与数据}
\subsection{数据说明}
本文使用中国区域统计年鉴2006-2013年的数据
\footnote{特别感谢何西龙师兄、刘晨冉师兄,以及财政学前沿课堂的其他同学所做出的数据整理的工作}。整理后数据共包括了中国31个省级行政单位(不包括海南省与直辖市)下的318个地级市的经济社会等变量的面板数据。
\subsection{计量模型与变量设定}
本文主要探究政府规模受到开放程度的影响。计量模型可以简化如下:
$$ govsize_{it} = \alpha + \beta_1 \cdot open_{it} + \beta_2 \cdot ln(gdppc_{it}) + \beta_3 \cdot \ln(pop_{it}) + prov_{i} +c_i + \mu_{it}$$


$govsize$为政府规模,采用同年的地方政府预算支出与地区生产总值的比值作为代理变量;$open$代表开放程度,由进出口贸易总额与地区生产总值的比值作为代理变量。
同时引入控制变量人均GDP($gdppc$)的对数、人口数量($pop$)的对数,与省份固定效应$prov_i$和城市固定效用$c_i$。

\subsection{描述性统计}
本文主要变量的描述性统计如表1:
    % 描述性统计表1
    \begin{table}
        \begin{center}
            \caption{描述性统计}
            \begin{threeparttable}
            \begin{tabular}{lccccc} \hline \hline
                 & (1) & (2) & (3) & (4) & (5) \\
                变量名 & 观测值数量 & 平均数 & 标准差 & 最小值 & 最大值 \\ \hline
                 &  &  &  &  &  \\
                govsize & 2,539 & 0.201 & 0.201 & 0.0427 & 3.581 \\
                govedu & 2,521 & 0.0361 & 0.0294 & 0.000304 & 0.261 \\
                govedu1 & 2,521 & 0.187 & 0.0466 & 0.00154 & 0.366 \\
                govsec & 2,524 & 0.0256 & 0.0738 & 0.000307 & 2.724 \\
                govsec1 & 2,524 & 0.117 & 0.0556 & 0.00119 & 0.761 \\
                open & 2,369 & 0.0286 & 0.0504 & 2.17e-06 & 0.472 \\
                open1 & 2,368 & 0.0472 & 0.112 & -3.022 & 0.853 \\
                lngdppc & 2,531 & 10.06 & 0.710 & 7.921 & 12.19 \\
                lnpop & 2,510 & 5.661 & 0.838 & 2.104 & 7.080 \\
                 &  &  &  &  &  \\ \hline \hline
            \end{tabular}

            \begin{tablenotes}
                \item 说明:
                $govedu$为地方政府当年的教育预算支出占GDP的比重;$govedu1$为当年的教育预算支出占当年政府预算支出的比重。
                 $govsec$为地方政府当年的社会保障预算支出占GDP的比重; $govsec1$为地方政府当年的社会保障预算支出占政府预算支出的比重。$open$为进出口贸易总额与当年地区生产总值的比值,$open1$为进出口贸易总额与当年不含第三产业地区生产总值的比值。$lngdppc$为人均地区生产总值的自然对数,$lnpop$为人口总数的自然对数。
            \end{tablenotes}
            \end{threeparttable}
        \end{center}
    \end{table}


\section{实证分析}
    由于本文使用数据集的特殊性(短面板、截面相关性特别强),所以本文采取了不同的估计方式,来得到更加稳健的结果。
\subsection{截面估计}
    首先,本文使用了截面估计。通过对每个年度的数据进行独立的截面回归,
    \begin{equation}
        govsize_{it} = \alpha_t + \beta_{1t} \cdot open_{it} + \beta_{2t} \cdot ln(gdppc_{it}) + \beta_{3t} \cdot \ln(pop_{it}) + prov_{it} +c_{it} + \mu_{it}
    \end{equation}


    估计结果见表2:

    % 截面估计表2
    \newgeometry{left=1cm}
    \begin{table}
        \begin{center}
            \caption{截面回归结果}
                \begin{tabular}{lcccccccc} \hline \hline

                    % \small
                    % \multicolumn{9}{c}{截面回归结果} \\ 
                    % \hline
                     & (1) & (2) & (3) & (4) & (5) & (6) & (7) & (8) \\
                    VARIABLES & 2006 & 2007 & 2008 & 2009 & 2010 & 2011 & 2012 & 2013 \\ \hline
                     &  &  &  &  &  &  &  &  \\
                    open & 0.250*** & 0.272*** & 0.280** & 0.402** & 0.361** & 0.437*** & 0.455*** & 0.598*** \\
                     & (0.0792) & (0.0519) & (0.126) & (0.176) & (0.176) & (0.149) & (0.157) & (0.189) \\
                    lngdppc & -0.0989*** & -0.0661*** & -0.133*** & -0.162*** & -0.195*** & -0.197*** & -0.205*** & -0.200*** \\
                     & (0.00728) & (0.00459) & (0.0124) & (0.0134) & (0.0154) & (0.0130) & (0.0135) & (0.0175) \\
                    lnpop & -0.0428*** & -0.0182*** & -0.0509*** & -0.0631*** & -0.0804*** & -0.0707*** & -0.0677*** & -0.0639*** \\
                     & (0.00589) & (0.00342) & (0.00985) & (0.0106) & (0.0117) & (0.00989) & (0.0102) & (0.0126) \\
                    % anr & -0.0576 & 0.0421*** & 0.0629 & 0.109 & 0.0693 & 0.123 & -0.0718 & -0.0631 \\
                    %  & (0.0441) & (0.0120) & (0.0855) & (0.118) & (0.110) & (0.0829) & (0.0682) & (0.0943) \\
                    Constant & 1.401*** & 0.888*** & 1.732*** & 2.129*** & 2.588*** & 2.585*** & 2.903*** & 2.834*** \\
                     & (0.0823) & (0.0532) & (0.158) & (0.189) & (0.199) & (0.169) & (0.165) & (0.212) \\
                     &  &  &  &  &  &  &  &  \\
                    Observations & 307 & 182 & 299 & 302 & 311 & 309 & 307 & 312 \\
                     R-squared & 0.619 & 0.721 & 0.535 & 0.579 & 0.579 & 0.651 & 0.648 & 0.638 \\ 
                     \hline
                     \hline
                    % \multicolumn{9}{c}{ Standard errors in parentheses} \\
                    % \multicolumn{9}{c}{ *** p$<$0.01, ** p$<$0.05, * p$<$0.1} \\
                \end{tabular}
        \end{center}
    \end{table}
    \restoregeometry


        从截面回归结果可以发现,在各个截面上,人均GDP和人口数的增长对政府规模有负作用,而经济开放程度对政府规模有正向影响。而且这种影响,随着时间的推移,在不断地增大。
        为了保证结果的稳健性,同时进行了
        (1)使用进出口贸易总额与当年不含第三产业地区生产总值的比值作为核心解释变量;
        (2)使用开放程度的滞后项作为核心解释变量;
        (3)使用教育支出比例作为政府规模的代理变量。
        系数及各项的显著程度与表格相似,说明了结果的稳定性。
    \footnote{由于篇幅限制,结果不逐一列出。}

\subsection{普通面板估计}
    豪斯曼检验统计量大小为63.15\%,在1\%的水平上显著,说明存在固定效应。分别使用混合估计和双向固定效应估计,估计
    \begin{equation}
        govsize_{it} = \alpha + \beta_{1} \cdot open_{it} + \beta_{2} \cdot ln(gdppc_{it}) + \beta_{3} \cdot \ln(pop_{it}) + prov_{it} + \mu_{it}
    \end{equation}
    \begin{equation}    
        govsize_{it} = \alpha + \beta_{1} \cdot open_{it} + \beta_{2} \cdot ln(gdppc_{it}) + \beta_{3} \cdot \ln(pop_{it}) + prov_{it} +c_{i} +d_{t} + \mu_{it}
    \end{equation}
    
    表3为估计结果。
    
    % 面板估计
    \begin{table}
        \begin{center}
            \caption{普通面板估计结果}
            \begin{tabular}{lcc} \hline \hline
                 & (1) & (2) \\
                VARIABLES & pool P\_clus & FE P\_clus \\ \hline
                 &  &  \\
                open & 0.310*** & 0.338*** \\
                 & (0.0565) & (0.100) \\
                lngdppc & -0.0862*** & -0.0266 \\
                 & (0.00731) & (0.0217) \\
                lnpop & -0.0232*** & -0.0245 \\
                 & (0.00455) & (0.0270) \\
                anr & 0.0727*** &  \\
                 & (0.00556) &  \\
                % o.anr &  & - \\
                %  &  &  \\
                Constant & 1.071*** & 0.590 \\
                 & (0.0843) & (0.344) \\
                 &  &  \\
                Observations & 1,392 & 1,392 \\
                R-squared & 0.726 & 0.603 \\
                 Number of pref &  & 174 \\ \hline \hline
                \multicolumn{3}{c}{ Robust standard errors in parentheses} \\
                \multicolumn{3}{c}{ *** p$<$0.01, ** p$<$0.05, * p$<$0.1} \\
            \end{tabular}
        \end{center}
    \end{table}

    此时估计的结果同截面估计的结果产生了偏差,可能是由于数据的性质。检验结果证明存在截面相关和异方差性。考虑进行面板协整分析。

\subsection{面板协整}

    % 协整估计
    \begin{table}
        \begin{center}
        \caption{协整估计}
            \caption{协整估计结果}
            \begin{tabular}{lccccc} \hline \hline
             & (1) & (2) & (3) & (4) & (5) \\
            VARIABLES & pool P\_clus & FE P\_clus & MG & CMG & CMG \\ \hline
             &  &  &  &  &  \\
            open & 0.310*** & 0.338*** & -2.369 & -0.760 & -0.186 \\
             & (0.0565) & (0.100) & (1.601) & (1.340) & (1.439) \\
            lngdppc & -0.0862*** & -0.0266 & -0.0450*** & -0.0529** & -0.0776** \\
             & (0.00731) & (0.0217) & (0.0102) & (0.0220) & (0.0329) \\
            lnpop & -0.0232*** & -0.0245 &  &  &  \\
             & (0.00455) & (0.0270) &  &  &  \\
            anr & 0.0727*** &  &  &  &  \\
             & (0.00556) &  &  &  &  \\
            o.anr &  & - &  &  &  \\
             &  &  &  &  &  \\
            \_\_000007\_t &  &  & 0.0144*** &  & -0.00465 \\
             &  &  & (0.00151) &  & (0.00748) \\
            \_\_00000M\_govsize &  &  &  & 0.382 & 1.000 \\
             &  &  &  & (0.373) & (0.892) \\
            \_\_00000L\_open &  &  &  & -0.984 & -0.398 \\
             &  &  &  & (0.642) & (1.184) \\
            \_\_00000L\_lngdppc &  &  &  & 0.0818*** & 0.104*** \\
             &  &  &  & (0.0299) & (0.0391) \\
            Constant & 1.071*** & 0.590 & 0.552*** & -0.198 & -0.285 \\
             & (0.0843) & (0.344) & (0.0958) & (0.143) & (0.221) \\
             &  &  &  &  &  \\
            Observations & 1,392 & 1,392 & 1,392 & 1,392 & 1,392 \\
            R-squared & 0.726 & 0.603 &  &  &  \\
             Number of pref &  & 174 & 174 & 174 & 174 \\ \hline \hline
            \multicolumn{6}{c}{ Robust standard errors in parentheses} \\
            \multicolumn{6}{c}{ *** p$<$0.01, ** p$<$0.05, * p$<$0.1} \\
        \end{tabular}
        \end{center}
    \end{table}


\section{结论}
截面之间正向影响,长期之间的动态均衡影响是不显著的。

\renewcommand\refname{参考文献}
\bibliographystyle{apacite}%
\bibliography{bib1}
\end{document}
