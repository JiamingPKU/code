% 补充了结论
% 增加了养老金模型的结论
% 标点由英文切换为中文
% ----------------------------
% 又补充了一些结论

\section{结论}
    % \begin{enumerate}
    %     \item 通过时间序列分析的方法,我们发现,在过去的20年中, FOF资产总量的增长率可以通过一个ARMA(0,5)-GARCH(1,1)模型来很好地拟合,并且此模型具有良好的预测效果.
    %     \item 2007年以来,美国的养老金资产处于缓慢增长状态.其对数增长率近似为白噪声序列,而且其分布呈现为负向更为集中的厚尾分布.
    %     \item FOF基金市场和养老金市场之前存在协整关系.在长期均衡的关系下,FOF资产总量占养老金市场总量的15\%水平.FOF基金市场的发展,也受益于养老金市场规模的不断扩大.
    % \end{enumerate}

本文主要对FOF 市场总资产规模,以及其与养老金市场资产之间的关联进行了实证研究.

首先利用时间序列分析的方法,对过去20年FOF资产总量建模分析.在检验其增长率的平稳性后,并处理一个强影响点之后,建立了一个ARMA(0, 5) - GARCH(1,1)模型,很好地拟合了既有数据.该模型表明FOF基金资产处于持续增长状态,其增长率有着尖峰厚尾的特点. 此模型也对真实序列的趋势与波动产生了良好的预测.

接下来又利用2007-2016年季度数据,对美国养老金市场中的IRA+DC部分资产进行了建模分析.发现2007年以来,美国的养老金资产处于缓慢增长状态.其对数增长率近似为白噪声序列,而且其分布呈现为负向更为显著的厚尾分布.

最后,本文发现FOF基金市场和养老金市场之间存在协整关系.在长期均衡的关系下,FOF资产总量占养老金市场总量的15\%水平.FOF基金市场的发展,也受益于养老金市场规模的不断扩大.并通过建立误差修正模型,估计了上一期的不均衡误差对当期以38\%的比率进行了修正.

\clearpage





