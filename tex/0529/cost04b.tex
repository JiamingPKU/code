\documentclass[12pt,twoside]{article}
\usepackage{chadstyle}  % Loads my formatting 
\usepackage{tweaklist}   

%% Shortcut commands
\newcommand{\growth}[1]{\frac{\dot{#1}_t}{{#1}_t}}
\newcommand{\prtl}[2]{\frac{\partial #1}{\partial #2}}
\newcommand{\commentno}[2]{\underset{\mbox{#2}}{#1}}
\newcommand{\comment}[2]{\underbrace{#1}_{\mbox{#2}}}
\newcommand{\cn}[1]{\citet*{#1}}
\newcommand{\cnp}[1]{(\citealt{#1})}  % (Jones 2002)
\newcommand{\fignote}[2]{\begin{center}\parbox[c]{#1}{\footnotesize #2} \end{center}}
\newcommand{\tabnote}[2]{\begin{center}\parbox[c]{#1}{\footnotesize #2} \end{center}}
\newcommand{\Prob}[1]{\mbox{\textnormal{Pr}} \, [#1] } 
\newcommand{\SubSubSection}[1]{\subsubsection{#1} \baselineskip=18pt}
\newcommand{\problem}[2]{\hspace{.1in} {{\color{ChadBlue}{\bf #1:} #2}}}
\newcommand{\proptitle}[1]{\color{ChadBlue} \textnormal{(#1):}}
\newtheorem{proposition}{\color{ChadGreen} Proposition}
\newcommand{\assume}[2]{{\bf{Assumption #1}} (#2)} 
\newcommand{\clr}[1]{{\color{ChadBlue} #1}}
\newcommand{\clrg}[1]{{\color{ChadGreen} #1}}
\newcommand{\Proof}[2]{\newline {\hspace{-\parindent} {\color{ChadGreen}\bf Proof of Proposition}~\ref{#1}.}
{\color{ChadBlue} #2} \vspace{.1in}}

% Adjust spacing in itemize/enumerate; see tweaklist.sty
\renewcommand{\enumhook}{\setlength{\topsep}{2pt}%
  \setlength{\itemsep}{0pt}}
\renewcommand{\itemhook}{\setlength{\topsep}{2pt}%
  \setlength{\itemsep}{0pt}}


\begin{document}
\bibliographystyle{aernobold}

%%%%%%%%%%%%%%%%%%%%%%%%%%%%%%%%%%%%%%%%%%%%%%%%%%%%%%%%%%%%%%%%%
% TITLE PAGE
%%%%%%%%%%%%%%%%%%%%%%%%%%%%%%%%%%%%%%%%%%%%%%%%%%%%%%%%%%%%%%%%%

%\begin{singlespacing}
\begin{spacing}{0.9}
\begin{titlepage}

%Two authors: \author{Matthias Doepke\\{\small UCLA, NBER, CEPR, and IZA}
%\and Mich\`{e}le Tertilt\\
%{\small Stanford University, NBER, and CEPR}}

\title{The Costs of Economic Growth}
\runningheads{Charles I. Jones}{The Costs of Economic Growth}


\author{\large \href{http://elsa.berkeley.edu/~chad}{Charles I.
    Jones}\thanks{I am grateful to the National Science Foundation for
    financial support and to the Stanford GSB for hosting me during this
    research.}\\ { U.C. Berkeley and NBER}
%{\normalsize \url{http://www.econ.berkeley.edu/~chad}}
}

\date{\small \today \ -- Version 0.4\\ {\it Preliminary}}
\maketitle
\thispagestyle{empty}

%\clearpage
\vspace{-0.3in}

\begin{abstract}

The benefits of economic growth are widely touted in the literature. But
what about the costs? Pollution, nuclear accidents, global warming, the
rapid global transmission of disease, and bioengineered viruses are just
some of the dangers created by technological change. How should these be
weighed against the benefits, and in particular, how does the
recognition of these costs affect the theory of economic growth? This
paper shows that taking these costs into account has first-order
consequences for economic growth. The rising value of life associated
with standard utility functions generates a conservative bias to
technological change, significantly slowing the optimal rate of economic
growth.

\end{abstract}

\end{titlepage}
\end{spacing}
%\end{singlespacing}


\begin{quote}
{\it  
Certain events quite within the realm of possibility, such as a major
asteroid collision, global bioterrorism, abrupt global warming --- even
certain lab accidents--- could have unimaginably terrible consequences
up to and including the extinction of the human race...  I am not a
Green, an alarmist, an apocalyptic visionary, a catastrophist, a Chicken
Little, a Luddite, an anticapitalist, or even a pessimist.  But... I
have come to believe that what I shall be calling the ``catastrophic
risks'' are real and growing...}
\end{quote} \hspace{2.8in} --- Richard A. \citet[p. v]{Posner2004}

%%%%%%%%%%%%%%%%%%%%%%%%%%%%%%%%%%%%%%%%%%%%%%%%%%%%%%%%%%%
\section{Introduction}
%%%%%%%%%%%%%%%%%%%%%%%%%%%%%%%%%%%%%%%%%%%%%%%%%%%%%%%%%%%


In October 1962, the Cuban missile crisis brought the world to the brink
of a nuclear holocaust. President John F. Kennedy put the chance of
nuclear war at ``somewhere between one out of three and even.'' The
historian Arthur Schlesinger, Jr., at the time an adviser of the
President, later called this ``the most dangerous moment in human
history.''\footnote{For these quotations, see \cite[p. 26]{Rees2003}.}
What if a substantial fraction of the world's population had been killed
in a nuclear holocaust in the 1960s? In some sense, the overall cost of
the technological innovations of the preceding 30 years would then seem
to have outweighed the benefits.

While nuclear devastation represents a vivid example of the potential
costs of technological change, it is by no means unique. The benefits
from the internal combustion engine must be weighed against the costs
associated with pollution and global warming. Biomedical advances have
improved health substantially but made possible weaponized anthrax and
lab-enhanced viruses. The potential benefits of nanotechnology stand
beside the ``grey goo'' threat that a self-replicating machine could
someday spin out of control. Experimental physics has brought us x-ray
lithography techniques and superconductor technologies but also the
remote possibility of devastating accidents as we smash particles
together at ever higher energies. These and other technological dangers
are detailed in a small but growing literature on so-called
``existential risks''; \cn{Posner2004} is likely the most familiar of
these references, but see also \cn{Bostrom2002}, \cn{Joy2002},
\cn{Overbye2008}, and \cn{Rees2003}.

Technologies need not pose risks to the existence of humanity in order
to have costs worth considering. New technologies come with risks as
well as benefits. A new pesticide may turn out to be harmful to
children. New drugs may have unforeseen side effects. Marie Curie's
discovery of the new element radium led to many uses of the
glow-in-the-dark material, including a medicinal additive to drinks and
baths for supposed health benefits, wristwatches with luminous dials,
and as makeup --- at least until the dire health consequences of
radioactivity were better understood. Other examples of new products
that were intially thought to be safe or even healthy include
thalidomide, lead paint, asbestos, and cigarettes.

The benefits of economic growth are truly amazing and have made enormous
contributions to welfare. However, this does not mean there are not also
costs. How does this recognition affect the theory of economic growth?
This paper shows that taking these costs into account has first-order
consequences. In particular, the rising value of life associated with
standard utility functions generates a conservative bias to
technological change, significantly slowing the optimal rate of economic
growth.


\section{The Model}

At some level, this paper is about speed limits. You can drive you car
slowly and safely, or fast and recklessly. Similarly, a key decision the
economy must make is to set a safety threshold: researchers can
introduce many new ideas without regard to safety, or they can select a
very tight safety threshold and introduce fewer ideas each year, slowing
growth to some extent.

The model below is a standard idea-based growth model, along the lines
of \cn{Romer90} and \cn{JonesRND}. Researchers introduce new varieties
of intermediate goods, and the economy's productivity is increasing in
the number of varieties. The key change relative to standard models is
that each variety $i$ also comes with a danger level, $z^i$. Some ideas
are especially dangerous (nuclear weapons or lead paint) and have a high
value of $z^i$, while other ideas are relatively harmless and have a low
$z^i$. The mortality rate in the economy depends on the values of the
$z^i$ that are consumed as well as on the amount consumed.

In the equilibrium allocation we study, firms that sell dangerous
products must pay a fee for each person they kill, and this fee is a
price determined in equilibrium.




\subsection{The Economic Environment}

The economy features three types of goods: consumption goods (which come
in a range of varieties), ideas, and people. People and ideas are the
two key factors of production, combining to produce the consumption
goods and new ideas.

At any point in time, a variety of consumption goods indexed by $i$ on
the interval $[0,A_t]$ are available for purchase.  We could define
utility directly over this variety of goods, but for the usual reasons,
it is easier to handle the aggregation on the production side.  Hence,
we assume these varieties combine in a CES fashion to produce a single
aggregate consumption good:
\begin{equation}
\label{eq:C}
\clr{ C_t = \left(\int_0 ^{A_t} X_{it} ^\theta di \right)^{1/\theta},
  \ \ \ \ \theta>1 }
\end{equation}

New varieties (ideas) are produced by researchers. If $L_{at}$ units of
labor are used in research with a current stock of knowledge $A_t$, then
research leads to the discovery of $\alpha L_{at} ^\lambda A_t ^\phi$
new varieties. This technology for producing new ideas is similar to
\cn{JonesRND}.

What's novel here is that each new variety $i$ is also associated with a
danger level, $z^i$. This danger level is drawn from a distribution with
cdf $F(z)$ and is observed as soon as the variety is discovered.
Researchers decide whether or not to complete the development of a new
variety after observing its danger level. Given that varieties are
otherwise symmetric, this leads to a cutoff level $z_t$: varieties with
a danger level below $z_t$ get implemented, whereas more dangerous
varieties do not. $z_t$ is a key endogenous variable determined within
the model. The fraction $F(z_t)=\Prob{z^i \leq z_t}$ of candidate
varieties get implemented, so the additional number of new varieties
introduced at any point in time is
\begin{equation}
\label{eq:Adot}
\clr{ \dot{A}_t = \alpha F(z_t) L_{at} ^\lambda A_t ^\phi, \ \ \ \ A_0 \mbox{ given}.}
\end{equation}

One unit of labor can produce one unit of any existing variety, and
labor used for different purposes cannot exceed the total amount
available in the economy, $N_t$:
\begin{equation}
\label{eq:RC}
\clr{ \int_0 ^{A_t} X_{it} di + L_{at} \leq N_t. }
\end{equation}
This total population is assumed to grow over time according to
\begin{equation}
\label{eq:Ndot}
\clr{ \dot{N}_t = (\bar{n} - \delta_t) N_t, \ \ \ \ N_0 \mbox{ given}. }
\end{equation}
The parameter $\bar{n}$ captures exogenous fertility net of mortality
unrelated to technological change.

Mortality from technological danger is denoted $\delta_t$. In principle,
it should depend on the amount of each variety consumed and the danger
associated with each variety, and it could even be stochastic (nuclear
weapons are a problem only if they are used). There could also be timing
issues: the use of fossil fuels today creates global warming that may be
a problem in the future.

These issues are interesting and could be considered in future work. To
keep the present model tractable, however, we make some simplifying
assumptions in determining mortality. In particular, all of the deaths
associated with any new technology occur immediately when that
technology is implemented, and the death rate depends on average
consumption across all varieties.  That is,
\begin{equation}
\label{eq:dlta}
\clr{ \delta_t = \bar{\delta} \dot{A}_t x_t \Gamma(z_t),}
\end{equation}
where $x_t \equiv \int_0 ^{A_t} X_{it}/N_t di / A_t$ is the average
amount consumed of each variety and $\Gamma(z_t) \equiv E[z^i | z^i \leq
  z_t] = \int_0 ^{z_t} z dF(z) / F(z_t)$ is the average mortality rate
associated with those new varieties. The mortality rate $\delta_t$,
then, is the product of the per capita quantity of new varieties
consumed, $\dot{A}_t x_t$, and their average mortality rate.

Individuals care about expected utility, where the expectation is
taken with respect to mortality.  Let $S_t$ denote the probability a
person survives until date $t$ conditional on being alive at date 0.
Expected utility is given by
\begin{equation}
\clr{ U= \int_0 ^\infty e^{-\rho t} u(c_t) S_t dt,}
\end{equation}
where
\begin{equation}
\clr{ \dot{S}_t = -\delta_t S_t, \ \ \ S_0=1.}
\end{equation}
Finally, we assume that flow utility $u(c)$ is
\begin{equation}
\label{eq:u}
\clr{ u(c_t) = \bar{u} + \frac{c_t ^{1-\gamma}}{1-\gamma}, \ \ \ \ c_t
  \equiv C_t/N_t, \ \ \bar{u}>0.}
\end{equation}
The key properties of this utility function are discussed next.

\subsection{Flow Utility and the Value of Life}

The cost of dying is the loss of future periods of utility. The flow
lost in year $t$ depends directly on $u(c_t)$. Figure~\ref{fig:u} shows
this flow utility for the CRRA formulation used in this paper,
equation~\eqref{eq:u}, for the special case in which $\gamma>1$.  This
case turns out to be especially interesting in what follows.
\begin{figure}[tp]
\caption{Flow Utility $u(c)$ for $\gamma>1$}
\label{fig:u}
%\centering \includegraphics[width=3.5in]{utilitygraph.ps} 
\fignote{4.5in}{Note: Flow utility is bounded for $\gamma>1$. If
  $\bar{u}=0$, then flow utility is negative and dying is
  preferred to living.}
\end{figure} 
There are two points to notice in this graph. First, if $\bar{u}=0$,
then flow utility is negative. Since we have (implicitly) normalized the
utility of death at zero in writing lifetime utility, utility would be
maximized by never living in this case. Hence $\bar{u}>0$ is required
for this model to make sense.

Second, flow utility is bounded for $\gamma>1$. Marginal utility goes to
zero very quickly for these preferences. Eating more sushi on a given
day when one is already eating sushi for breakfast, lunch, and dinner
has very low returns. Instead, preserving extra days of life on which to
eat sushi is the best way to increase utility.

This point can also be made with the algebra. Valued in units of
consumption instead of utils and expressed as a ratio to current
consumption, the flow value of a year of life is
\begin{equation}
 \clr{ \frac{u(c_t)}{u'(c_t) c_t} = \bar{u} c_t ^{\gamma-1} -
   \frac{1}{1-\gamma} }
\end{equation}
For $\gamma>1$, the value of life rises faster than consumption; this is
the essential mechanism that leads the economy to tilt its allocation
away from consumption growth and toward preserving life in the model.
This point is more general than the particular utility function assumed
here. For example, any bounded utility function will deliver this
result, as will log utility.\footnote{For log preferences,
  $u(c)=\bar{u}+\log c$. Because $u'(c)c=1$, the value of a year of life
  in consumption units is just $u(c)$ itself, which increases without
  bound in consumption.}

\subsection{A Rule of Thumb Allocation}

Given the symmetry of $X_{it}$, there are two nontrivial allocative
decisions that have to be made in this economy at each date. First is
the allocation of labor between $L_{at}$ and $X_{it}$. Second is the key
tradeoff underlying this paper, the choice of the safety threshold
$z_t$.  A high cutoff for $z_t$ implies that more new ideas are
introduced in each period but it also means a higher mortality rate.
This is the model's analog to driving fast and recklessly instead of
slowly and safely.

In the next main section, we will let markets allocate resources and
study an equilibrium allocation. To get a sense for how the model works,
however, it is convenient to begin first with a simple rule of thumb
allocation. For this example, we assume the economy puts a constant
fraction $\bar{s}$ of its labor in research and allocates the
remainder symmetrically to the production of the consumption goods. In
addition, we assume the safety cutoff is constant over time at
$\bar{z}$.

Let $g_x$ denote the growth rate of some variable $x$ along a balanced
growth path. Then, we have the following result (proofs for this and
other propositions are given in Appendix~\ref{app:proofs}):
\begin{proposition}
\proptitle{\hyperlink{proof:rule}{BGP under the Rule of Thumb Allocation}}
\label{prop:rule} \hypertarget{prop:rule}{}
Under the rule of thumb allocation, there exists a balanced growth path
such that $g_c=\sigma g_A$ and
\begin{equation}
\label{eq:drule}
\delta^* = \bar{\delta} g_A (1-\bar{s}) \Gamma(\bar{z})
\end{equation}
\begin{equation}
\label{eq:gNrule}
 g_N = \bar{n}-\delta^*
\end{equation}
\begin{equation}
\label{eq:gArule}
g_A =\frac{\lambda (\bar{n}-\delta^*)}{1-\phi} = \frac{\lambda \bar{n}}{1-\phi+\lambda \bar{\delta}
  (1-\bar{s}) \Gamma(\bar{z})}.  
\end{equation}
\end{proposition}

Along the balanced growth path, the mortality rate is constant and
depends on (a) how fast the economy grows, (b) the intensity of
consumption, and (c) the danger threshold. As in \cn{JonesRND}, the
steady-state growth rate is proportional to the rate of population
growth. However, the population growth rate is now an endogenous
variable because of endenous mortality.  For example, an increase in
research intensity $\bar{s}$ will reduce the steady-state mortality rate
(a lower consumption intensity) and therefore increase the long-run
growth rate.

The effect of changing the danger threshold $\bar{z}$ is more subtle
and is shown graphically in Figure~\ref{fig:rule}.
\begin{figure}[tp]
\caption{Growth under the Rule of Thumb Allocation}
\label{fig:rule}
%\centering \includegraphics[width=3.5in]{rule2.ps} 
\fignote{4.5in}{Note: There is a medium-run tradeoff between growth and
  technological danger, but no long-run tradeoff.  In the long run,
  safer choices lead to faster net population growth and therefore
  faster consumption growth.}
\end{figure} 
As emphasized earlier, there is indeed a basic tradeoff in this model
between growth and safety.  Over the first 300+ years in the example,
the safer choice of $\bar{z}$ leads to slower growth as researchers
introduce fewer new varieties.  However, this tradeoff disappears in the
long run because the growth rate itself depends on population growth.  A
safer technology choice reduces the mortality rate, raises the
population growth rate, and therefore raises consumption growth in the
long run.

%% Two last remarks are worth noting before we turn to an equilibrium
%% allocation in this model.  First, by setting the danger threshold
%% sufficiently low, we can make the growth rate arbitrarily close to
%% $\lambda \bar{n}/(1-\phi)$, which is the familiar semi-endogenous growth
%% rate in \cn{JonesRND}.  Second, for any threshold $\bar{z}>0$, there is
%% no ``knife-edge'' problem at $\phi=1$.  This is because of the feedback
%% between technological change and mortality.  In standard idea-based
%% growth models, the condition $\phi=1$ leads to explosive growth in the
%% presence of population growth.  Here, however, 


\section{A Competitive Equilibrium with Patent Buyouts}

The rule of thumb allocation suggests that this model will deliver a
balanced growth path with an interesting distinction between the
medium-run and long-run tradeoffs between growth and safety.  Moreover,
the model features endogenous growth in the strong sense that changes in
policy can affect the long-run growth rate.  Somewhat surprisingly,
neither of these results will hold in the competitive equilibrium, and
our rule of thumb allocation turns out not to be a particularly good
guide to the equilibrium dynamics of the competitive equilibrium.

\subsection{An Overview of the Equilibrium}

A perfectly competitive equilibrium will not exist in this model because
of the nonrivalry of ideas \citep{Romer90}. Instead of following Romer
and introducing imperfect competition, we use a mechanism advocated by
\cn{Kremer98}. That is, we consider an equilibrium in which research is
funded entirely by ``patent buyouts'': the government in our model purchases new
ideas at a price $P_{at}$ and makes the designs publicly available at no
charge. The motivation for this approach is largely technical: it
simplifies the model so it is easier to understand. However, there is
probably some interest in studying this institution in its own right.

The other novel feature of this equilibrium is that we introduce a
competitive market for mortality: idea producers pay a price $v_t$ for
every person they kill, and households ``sell'' their mortality as if
survival were a consumer durable. This market bears some resemblence to
one that emerges in practice through the legal system of torts and
liabilities.

In equilibrium, these two institutions determine the key allocations.
Patent buyouts pin down the equilibrium amount of research and the
mortality market pins down the danger cutoff.

\subsection{Optimization Problems}

The equilibrium introduces three prices: a wage $w_t$, the price of
mortality $v_t$, and the price of new ideas $P_{at}$. The equilibrium
then depends on three optimization problems.

First, a representative household supplies a unit of labor, chooses how
much of her life to sell in the mortality market, pays a lump-sum tax
$\tau_t$, and eats the proceeds. Our timing assumption is that mortality
is realized at the end of the period, after consumption occurs.\vspace{.1in}

\hypertarget{HH}{
\problem{HH Problem}{Given $\{w_t, v_t, \tau_t\}$, the representative
household solves
\[ \max_{\{\delta_t ^h\}} \int_0 ^\infty e^{-\rho t} u(c_t) S_t dt \]
\begin{center}
s.t.  $c_t = w_t + v_t \delta_t ^h - \tau_t$ and $\dot{S}_t = -\delta_t ^h S_t$.
\end{center}}}

\vspace{-.1in} Next, a representative firm in the perfectly competitive
market for the final good (FG) solves the following profit maximization
problem:\vspace{.1in}

\hypertarget{FG}{}
\problem{FG Problem}{At each date $t$, given $w_t$ and $A_t$, 
\[ \max_{\{X_{it}\}} \left( \int_0 ^{A_t} X_{it} ^\theta di \right)^{1/\theta} - w_t
\int_0 ^{A_t} X_{it} di. \] }

\vspace{-.1in}
Finally, a representative research firm produces ideas in the perfectly
competitive idea sector and chooses a cutoff danger level $z_t$ based on
the price of mortality.  The research firm sees constant returns to idea
production at productivity $\alpha_t$, so the effects associated with
$\lambda<1$ and $\phi \neq 0$ are external:\vspace{.1in}

\hypertarget{RD}{}
\problem{R\&D Problem}{At each date $t$, given $P_{at}, w_t, v_t, x_t, \alpha_t$, 
\[ \max_{L_{at},z_t} P_{at} \alpha_t F(z_t) L_{at} - w_t L_{at} - v_t
\delta_t N_t  \ \mbox{ s.t } \ \delta_t = \bar{\delta} x_t \alpha_t
F(z_t) L_{at} \Gamma(z_t).  \]}

\vspace{-.2in}
\subsection{Defining the Competitive Equilibrium}

The competitive equilibrium in this economy solves the optimization
problems given in the previous section and the relevant markets clear.
The only remaining issue to discuss is the government purchase of ideas.
We've already assumed the government pays a price $P_{at}$ for any idea
and releases the design into the public domain. We assume this is the
only option for researchers --- there is no way to keep ideas secret and
earn a temporary monopoly profit. As discussed above, the reason for
this is to keep the model simple; nothing would change qualitatively if
we introduced monopolistic competition, either through secrecy or
patents.

In addition, we assume the idea purchases are financed with lump-sum
taxes on households and that the government's budget balances in each
period. Moreover, we assume the government sets the price at which ideas
are purchased so that total purchases of ideas are a constant proportion
$\beta$ of aggregate consumption; we will relax this assumption later.

The formal definition of the equilibrium allocation follows:
{\color{ChadBlue}
\begin{quote}
{\bf {Definition}}
A {\it CE with public support for R\&D} consists of quantities $\{c_t,\delta_t ^h,
X_{it}, A_t, L_{at}, N_t, \tau_t, \delta_t, z_t, \alpha_t, x_t \}$ and
prices $\{w_t,P_{at},v_t\}$ such that
\begin{enumerate}
\item $\{c_t,\delta_t ^h\}$ solve the \hyperlink{HH}{HH Problem}.
\item $X_{it}$ solve the \hyperlink{FG}{FG Problem}.
\item $L_{at},z_t,\delta_t,A_t$ solve the \hyperlink{RD}{R\&D Problem}.
\item $w_t$ clears the labor market: $\int_0 ^{A_t} X_{it}di+L_{at}=N_t$.
\item $v_t$ clears the mortality market: $\delta_t ^h=\delta_t$.
\item The government buys ideas: $P_{at} \dot{A}_t = \beta c_t N_t$.
\item Lump sum taxes $\tau_t$ balance the budget: $\tau_t =P_{at}
  \dot{A}_t / N_t$.
\item Other conditions:
$\dot{N}_t = (\bar{n} -\delta_t)N_t$,
$\alpha_t = \alpha L_{at} ^{\lambda-1} A_t ^\phi$, and 
$x_t \equiv \frac{1}{A_t} \int_0 ^{A_t} X_{it} di / N_t$.
\end{enumerate}
\end{quote}}

\subsection{The Benchmark Case}

The equilibrium behavior of the economy depends in important ways on a
few parameters. We specify a benchmark case that will be studied in
detail, and then in subsequent sections consider the effect of deviating
from this benchmark. In specifying the benchmark, it is helpful to note
that in equilbrium $c_t = A_t ^\sigma (1-s_t)$, where $\sigma \equiv
\frac{1-\theta}{\theta}$ is the elasticity of consumption with respect
to the stock of ideas. The benchmark case is then given by

\vspace{.1in} 
\hypertarget{ASSUME}{} \clr{
\assume{A.}{Benchmark Case} 
Let $\eta \equiv \lim_{z \rightarrow 0} F'(z) z/F(z)$.  Assume
\begin{enumerate} 
\addtolength{\itemsep}{-\baselineskip}
\setlength{\itemindent}{2em}
\item[A1.] Finite elasticity of $F(z)$ as $z \rightarrow 0$: $\ \eta \in (0,\infty)$  \\
\item[A2.] Rapidly declining marginal utility of consumption: $\ \gamma>1$ \\
\item[A3.] Knowledge spillovers are not too strong: $\ \phi<1+\eta \sigma (\gamma-1)$.
\end{enumerate}
} \vspace{.1in}
{\noindent 
We will discuss the nature and role of each of these assumptions in more
detail as we develop the results. The least familiar assumption is A1,
but note that both the exponential and the Weibull distributions have
this property. In contrast, the lognormal and Fr\'{e}chet distributions
have an infinite elasticity in the limit as $z$ goes to zero. This has
interesting implications that we will explore in detail.}

\subsection{The Equilibrium Balanced Growth Path}

We are now ready to solve for the equilibrium in this growth model with
dangerous technologies.  In particular, we can characterize a balanced
growth path:

\begin{proposition}
\proptitle{\hyperlink{proof:CE}{Equilibrium Balanced Growth}}
\label{prop:CE} \hypertarget{prop:CE}{}
Under Assumption A,
the competitive equilibrium exhibits an asymptotic balanced growth path
as $t \rightarrow \infty$ such that
\begin{equation}
s_t \rightarrow \frac{\beta}{1+\beta+\eta} 
\end{equation}
\begin{equation}
z_t \rightarrow 0 \mbox{ (and therefore } \delta_t \rightarrow 0
\mbox{)} 
\end{equation}
\begin{equation}
\label{eq:gz} \dot{z}_t/z_t \rightarrow g_z = - (\gamma-1) g_c 
\end{equation}
\begin{equation}
\dot{c}_t/c_t \rightarrow g_c=\sigma g_A \equiv \frac{\lambda \sigma
  \bar{n}}{1-\phi+\eta \sigma (\gamma-1)}
\end{equation}
\begin{equation}
\label{eq:v}
u'(c_t) v_t \rightarrow \frac{\bar{u}}{\rho}. 
\end{equation}
\end{proposition}


The somewhat surprising result that emerges from the equilibrium under
\hyperlink{ASSUME}{Assumption A} is that mortality and the danger
threshold, rather than being constant in steady state, decline at a
constant exponential rate.  Technological change becomes increasingly
conservative over time, as an increasing fraction of possible new ideas
are rejected because they are too dangerous.

The consequence of this conservative bias in technological change is no
less surprising: it leads to a slowdown in steady-state growth. There
are several senses in which this is true, and these will be explored as
the paper goes on. But two are evident now. First, the negative growth
rate of $z_t$ introduces a negative trend in TFP growth for the idea
sector, other things equal. Recall that the idea production function is
$\dot{A}_t = \alpha F(z_t) L_{at} ^\lambda A_t ^\phi$. $z_t$ declines at
the rate $(\gamma-1)g_c$, ultimately getting arbitrarily close to zero.
By Assumption A1, the elasticity of the distribution $F(z)$ at zero is
finite and given by $\eta$, so $F(z_t)$ declines at rate $\eta
(\gamma-1) g_c = \eta (\gamma-1) \sigma g_A$. 

The second way to see how this bias slows down growth is to focus
directly on consumption growth itself.  The steady-state rate of
consumption growth is 
\begin{equation}
\label{eq:gc}
\clr{ g_c=\frac{\lambda \sigma \bar{n}}{1-\phi+\eta \sigma (\gamma-1)}. }
\end{equation}
The last term in the denominator directly reflects the negative TFP
growth in the idea production function resulting from the tightening of
the danger threshold.  

That this slows growth can be seen by considering the following thought
experiment.  A feasible allocation in this economy is to follow the
equilibrium path until $z_t$ is arbitrarily small and then keep it
constant at this value.  This results in a mortality rate that is
arbitrarily close to zero, and the growth rate in this case will be
arbitrarily close to $\lambda \sigma \bar{n} / (1-\phi)$, which is
clearly greater than the equilibrium growth rate.  Rather than keep $z$
constant at a small level, the equilibrium continues to reduce the
danger cutoff, slowing growth.  Some numerical examples at the end of
this paper suggest that this slowdown can be substantial.

Of course, this raises a natural question: {\it Why} does the equilbrium
allocation lead the danger threshold to fall exponentially to zero? To
see the answer, first consider the economic interpretation of the
mortality price $v_t$. This is the price at which firms must compensate
households per unit of mortality that their inventions inflict. In the
terminology of the health and risk literatures, it is therefore equal to
the value of a statistical life (VSL).\footnote{Suppose the mortality
  rate is $\delta_t=.001$ and $v_t=$ \$1 million. In this example, each
  person receives \$1000 ($=v_t \delta_t$) for the mortality risk they
  face. For every thousand people in the economy, one person will die,
  and the total compensation paid out for this death will equal \$1
  million.}

Along the balanced growth path, the value of life satisfies
equation~\eqref{eq:v}:
\begin{equation}
\label{eq:vSS}
\clr{ u'(c_t) v_t \rightarrow \frac{\bar{u}}{\rho}. }
\end{equation}
This equation simply says that the value of life measured in utils is
asymptotically equal to the present discounted value of utility: as
consumption goes to infinity, flow utility converges to $\bar{u}$, so
lifetime utility is just $\bar{u}/\rho$.

Viewed in another way, this equation implies that the value of life
grows faster than consumption. Given our functional form assumption for
preferences, $u'(c_t)=c_t ^{-\gamma}$. So $c_t ^{-\gamma} v_t$ converges
to a constant, which means that $g_v \rightarrow \gamma g_c$. Because
$\gamma$ is larger than one, marginal utility falls rapidly, and the
value of life rises faster than consumption.

With this key piece of information, we can turn to the first-order
condition for the choice of $z_t$ in the \hyperlink{RD}{R\&D Problem}.
That first-order condition is
\begin{equation}
\label{eq:z}
\clr{ z_t = \frac{P_{at}}{v_t N_t \bar{\delta} x_t} = \frac{\beta
    c_t}{v_t \bar{\delta} g_{At} (1-s_t)}. }
\end{equation}
The first part of this equation says that the danger threshold $z_t$
equals the ratio of two terms. The numerator is related to the marginal
benefit of allowing more dangerous technologies to be used, which is
proportional the price at which the additional ideas could be sold. The
denominator is related to the marginal cost, which depends on the value
of the additional lives that would be lost.

The second equation in~\eqref{eq:z} uses the fact that $P_{at} \dot{A}_t
= \beta C_t$ to eliminate the price of ideas. This last expression
illustrates the key role played by the value of life. In particular, we
saw above that $v_t/c_t$ grows over time since $\gamma>1$; the value of
life rises faster than consumption. Because both $g_{At}$ and $1-s_t$
are constant along the balanced growth path, the rapidly rising value of
life leads to the exponential decline in $z_t$. More exactly, $v_t/c_t$
grows at rate $(\gamma-1) g_c$, so this is the rate at which $z_t$
declines, as seen in equation~\eqref{eq:gz}.

What is the economic intuition? Because $\gamma$ exceeds 1, flow utility
$u(c)$ is bounded and the marginal utility of additional consumption
falls very rapidly.  This leads the value of life to rise faster than
consumption. The benefit of using more dangerous technologies is that
the economy gets more consumption. The cost is that more people die.
Because the marginal utility of consumption falls so quickly, the costs
of people dying exceed the benefit of increasing consumption and the
equilibrium delivers a declining threshold for technological danger.
Safety trumps economic growth.


\subsection{Growth Consequences}

From the standpoint of growth theory, there are some interesting
implications of this model.  First, as we saw in the context of the
rule-of-thumb allocation, this model is potentially a fully endogenous
growth model.  Growth is proportional to population growth, but because
the mortality rate is endogenous, policy changes can affect the
mortality rate and therefore affect long-run growth, at least
potentially.

Interestingly, however, that is not the case in the equilibrium
allocation. Instead, the mortality rate trends to zero and is unaffected
by policy changes in the long run. Hence, the equilibrium allocation
features semi-endogenous growth, where policy changes have long-run
level effects but not growth effects. In particular, notice that the
patent buyout parameter $\beta$, which influences the long-run share of
labor going to research, is not a determinant of the long-run growth
rate in \eqref{eq:gc}. Moreover, the invariance of long-run growth to
policy is true even though a key preference parameter, $\gamma$,
influences the long-run growth rate.

Finally, it is interesting to consider the special case of $\phi=1$, so
the idea production function resembles that in \cn{Romer90}. In this
case,
\[
\clr{ \frac{\dot{A}_t}{A_t} = \alpha F(z_t) L_{at} ^\lambda. }
\]
Here, growth does not explode even in the presence of population growth
with $\phi=1$, as can be seen in equation~\eqref{eq:gc}. Instead, the
negative trend in $z_t$ and the fact that the mortality rate depends on
the growth rate conspire to keep growth finite.


\section{Extensions}

The crucial assumptions driving the results have been collected together
and labeled as \hyperlink{ASSUME}{Assumption A}. In this section, we
illustrate how things change when these assumptions are relaxed.
Briefly, there are two main findings. First, when we consider
distributions with an infinite elasticity at $z=0$, the concern for
safety exhibits an even more extreme technological bias: equilibrium
growth slows all the way to zero asymptotically. Second, we highlight
the role played by $\gamma>1$: if instead $\gamma<1$, then the
equilibrium allocation looks like the rule of thumb allocation,
selecting a constant danger threshold in steady state.


\subsection{Relaxing A1: Letting $\eta=\infty$}

Recall that $\eta$ is the elasticity of the danger distribution $F(z)$
in the limit as $z \rightarrow 0$.  Intuitively, this parameter plays an
important role in the model because $F(z)$ is the fraction of new ideas
that are used in the economy, and $z$ is trending exponentially to zero.
The term $\eta g_z = -\eta (\gamma-1) g_c$ therefore plays a key role in
determining the growth rate of ideas along the balanced growth path:
\begin{equation}
\label{eq:gA}
\clr{ g_A = \frac{\lambda \bar{n}}{1-\phi+\eta \sigma (\gamma-1)} }.
\end{equation}

Assumption A1 says that $\eta$ is finite. This is true for a number of
distributions, including the exponential ($\eta=1$), the Weibull, and
the gamma distributions. However, it is not true for a number of other
distributions. Both the lognormal and the Fr\'{e}chet distributions have
an infinite elasticity at zero, for example. Given that we have no prior
over which of these distributions is most relevant to our problem, it is
essential to consider carefully the case of $\eta=\infty$.

In fact, it is easy to get a sense for what will happen by considering
equation~\eqref{eq:gA}. As $\eta$ rises in this equation, the
steady-state growth rate of the economy declines. Intuitively, a 1\%
reduction in $z$ has a larger and larger effect on $F(z)$: an increasing
fraction of ideas that were previously viewed as safe are now rejected
as too dangerous. This reasoning suggests that as $\eta$ gets large, the
steady-state growth rate falls to zero, and this intuition is confirmed
in the following proposition:

\begin{proposition}
\proptitle{\hyperlink{proof:eta}{Equilibrium Growth with $\eta=\infty$}}
\label{prop:eta} \hypertarget{prop:eta}{}
Let Assumptions A2 and A3 hold, but instead of A1, assume $\eta=\infty$.
In the competitive equilibrium, as $t \rightarrow \infty$
\begin{enumerate}
%\setlength{\itemsep}{3pt}  \setlength{\parskip}{0pt}
%  \setlength{\parsep}{0pt} 
\item The growth rate of consumption falls to zero: $\dot{c}_t/c_t \rightarrow 0$
\item The level of consumption goes to infinity: $c_t \rightarrow \infty$
\item The technology cutoff, the mortality rate, and the share of labor
  devoted to research all go to zero: $z_t \rightarrow 0$, $\delta_t \rightarrow 0$, $s_t \rightarrow 0$.
\end{enumerate}
\end{proposition}

When $\eta=\infty$, the increasingly conservative bias of technological
change slows the exponential growth rate all the way to zero. However,
this does not mean that growth ceases entirely. Instead, the level of
consumption still rises to infinity, albeit at a slower and slower rate.

%Thick lower tail.  A 1\% reduction in z has a larger and larger effect,
%meaning that an increasing fraction of the remaining ideas get tossed
%out for being too dangerous.


\subsection{Relaxing A2: Assume $\gamma<1$}

The most important assumption driving the results in this paper is that
marginal utility diminishes quickly, in the sense that $\gamma>1$.  For
example, the value of a year of life in year $t$ as a ratio to
consumption is
\[ \clr{ 
  \frac{u(c_t)}{u'(c_t)c_t} = \bar{u} c_t^{\gamma-1}+
  \frac{1}{1-\gamma}}. 
\] 
For $\gamma>1$, this rises to infinity as consumption grows. But for
$\gamma<1$, it converges to $1/(1-\gamma)$: the value of life is
proportional to consumption. In this case, the elasticity of utility
with respect to consumption remains positive rather than falling to
zero, which keeps the value of life and consumption on equal footing.
The result is that the conservative bias of technological change
disappears: the economy features exponential growth in consumption with
a constant danger cutoff and a constant, positive mortality rate:

\begin{proposition}
\proptitle{\hyperlink{proof:gamma}{Equilibrium Growth with $\gamma<1$}}
\label{prop:gamma} \hypertarget{prop:gamma}{}
Let Assumptions A1 and A3 hold, but instead of A2, assume $\gamma<1$.
The competitive equilibrium exhibits an asymptotic balanced growth path
as $t \rightarrow \infty$ such that
\[ z_t \rightarrow z^* \in (0,\infty), \ \ \delta_t \rightarrow \delta^* \in (0,\infty)  \] 
\[ s_t \rightarrow \frac{\beta (1-\frac{\Gamma(z^*)}{z^*})}{1+\beta (1-\frac{\Gamma(z^*)}{z^*})} \]
\[ \growth{c} \rightarrow g_c=\frac{\lambda(\bar{n}-\delta^*)}{1-\phi}= \frac{\lambda \sigma
  \bar{n}}{1-\phi+\lambda \bar{\delta} (1-s^*) \Gamma(z^*)}\]
\[ \frac{v_t}{c_t} \rightarrow \frac{1}{1-\gamma} \cdot
\frac{1}{\rho+\delta^*-(1-\gamma) g_c} \]
\end{proposition}

For $\gamma<1$, the economy looks very similar to the rule of thumb
allocation; for example, compare the growth rate to that in
Proposition~\ref{prop:rule}. The economy features a constant danger
cutoff as well as endogenous growth: an increase in idea purchases by
the government (a higher $\beta$) will shift more labor into research,
lower the mortality rate, and increase the long-run growth
rate.\footnote{The intermediate case of log utility ($\gamma=1$)
  requires separate consideration. In this case, the technology cutoff
  $z_t$ still declines to zero, but this decline is slower than
  exponential. The long-run growth rate is then precisely back to the
  semi-endogenous growth case: $g_A = \lambda \bar{n}/(1-\phi)$. }


\subsection{Optimality}

In this section, we study the allocation of resources that maximizes a
social welfare function. There are three reasons for this. First, it is
important to verify that the declining danger threshold we uncovered in
the equilibrium allocation is not a perverse feature of our equilibrium.
Second, in the equilibrium allocation, individuals put no weight on the
welfare of future generations; it is a purely selfish equilibrium. It is
interesting to study the effect of deviations from this benchmark.
Finally, our equilibrium allocation employed a particular institution
for funding research: patent buyouts where spending on new ideas is in
constant proportion to consumption. This institution is surely special
(and not generally optimal), so it is important to confirm that it is
not driving the results.  The bottom line of this extension to an
optimal allocation is that all of the previous results hold up well.

In this environment with multiple generations, there is no indisputable
social welfare function. However, a reasonably natural choice that
serves our purposes is to treat flows of utility from different people
symmetrically and to discount flows across time at rate $\rho$. This
leads to the following definition of an optimal allocation:

{\color{ChadBlue}
\begin{quote}
{\bf {Definition}}
An {\it optimal allocation} in this economy is a time path for $\{s_t,
z_t\}$ that solves
\[ \max_{\{s_t,z_t\}} \int_0 ^\infty e^{-\rho t} N_t
u(c_t) dt \ \ \mbox{ subject to}\]
\[ c_t =A_t ^\sigma (1-s_t) \]
\[ \dot{A}_t = \alpha F(z_t) s_t ^\lambda N_t ^\lambda A_t ^\phi \]
\[ \dot{N}_t = (\bar{n}-\delta_t) N_t \]
\[ \delta_t = \bar{\delta} \alpha s_t ^\lambda N_t ^\lambda A_t
^{\phi-1} (1-s_t) \int_0 ^{z_t} zf(z)dz. \]
\end{quote}}

The optimal allocation can then be characterized as follows.

\begin{proposition}
\proptitle{\hyperlink{proof:swf}{Optimal Balanced Growth}}
\label{prop:swf} \hypertarget{prop:swf}{}
Under Assumption A and the optimal allocation, the economy exhibits an
asymptotic balanced growth path as $t \rightarrow \infty$ such that
\[ \frac{s_t}{1-s_t} \rightarrow \frac{\lambda \sigma g_A}
{(1+\eta)(\rho-\bar{n}+(\gamma-1)g_c) +(1-\phi)g_A - \eta \sigma g_A} \]
\[ z_t \rightarrow 0 \mbox{ (and therefore } \delta_t \rightarrow 0
\mbox{)}  \]
\[ \dot{z}_t/z_t \rightarrow - (\gamma-1) g_c  \]
\[ \dot{c}_t/c_t \rightarrow g_c, \]
where $g_c$ and $g_A$ are the same as in the competitive equilibrium.
\end{proposition}

The key properties of the competitive equilibrium carry over into the
optimal allocation.  In particular, the danger threshold declines
exponentially to zero at the rate $(\gamma-1) g_c$, and this
technological bias slows the growth rate of the economy.  The long-run
growth rate is the same as in the equilibrium allocation.


\section{Numerical Examples}

We now report a couple of numerical examples to illustrate how this
economy behaves along the transition path. We make some attempt to
choose plausible parameter values and to produce simulations that have a
``realistic'' look to them. However, the model abstracts from a number
of important forces shaping economic growth and mortality, so the
examples should not be taken too literally. Mainly, they will illustrate
the extent to which growth can be slowed by concerns about the dangers
of certain technologies.

The first example features sustained exponential growth ($\eta<\infty$).
The second assumes a distribution $F(z)$ with an elasticity that rises
to infinity as $z$ falls to zero. According to
Proposition~\ref{prop:eta}, this example exhibits a growth rate that
declines to zero, even though consumption itself rises indefinitely.

\begin{table}[tp]
\caption{Benchmark Parameter Values}
\label{tab:params} %\vspace{.15in}
\begin{tabular*}{\textwidth}{@{\extracolsep{\fill}}cl} \hline
 Parameter Value & Comment   \\ \hline
$\gamma=1.5$ & Slightly more curvature than log utility \\
$\eta=1$ & $F(z)$ is an exponential distribution  \\
$\beta=0.02$ & Government spends 2\% of consumption on ideas  \\
$\lambda=1$, $\phi=1/2$ & Idea production function: $\dot{A}_t = \alpha F(z_t) A_t ^{1/2} L_{at}$  \\
$\bar{n}=.01$ & Long-run population growth rate   \\
$\sigma=2$ & Elasticity of consumption wrt ideas  \\
$\rho=.05$ & Rate of time preference   \\
$\bar\delta=50$ & Mortality rate intercept   \\
\hline
\end{tabular*}
\tabnote{5.5in}{Note: These are the baseline parameter values for the
  numerical examples.}  
\end{table}

\subsection{Benchmark Example}

The basic parameterization of the benchmark case is described in
Table~\ref{tab:params}. For the curvature of marginal utility, we choose
$\gamma=1.5$; large literatures on intertemporal choice \cnp{Hall:IES},
asset pricing \cnp{Lucas:risk}, and labor supply \cnp{Chetty2006}
suggest that this is a reasonable value. For $F(z)$, we assume an
exponential distribution so that $\eta=1$; we also assume this
distribution has a mean of one. We set $\beta=.02$: government spending
on ideas equals 2\% of aggregate consumption. For the idea production
function, we choose $\lambda=1$ and $\phi=1/2$, implying that in the
absence of declines in $z_t$, the idea production function itself
exhibits productivity growth. Finally, we assume a constant population
growth rate of 1\% per year. The other parameter values are relatively
unimportant and are shown in the table. Other reasonable choices for
parameter values will yield similar results qualitatively. The model is
solved using a reverse shooting technique, discussed in more detail in
Appendix~\ref{app:solve}.\hypertarget{APPB}{}

Figure~\ref{fig:dynamics} shows an example of the equilibrium dynamics
that occur in this economy for the benchmark case.
\begin{figure}[t!p]
\caption{Equilibrium Dynamics: Benchmark Case}
\label{fig:dynamics}
%  \centering \includegraphics[width=3.5in]{cost2b.ps} 
\fignote{4.5in}{Note: Simulation results for the competitive equilibrium
using the parameter values from Table~\ref{tab:params}.  Consumption
growth settles down to a constant positive rate, substantially lower
than what is feasible.  The danger threshold and mortality rate
converge to zero.}
\end{figure} 
The economy features a steady-state growth rate of per capita
consumption of 1.33\%. This constant growth occurs while the danger
threshold and the mortality rate decline exponentially to zero; both
$z_t$ and $\delta_t$ grow at -0.67\%.

Several other features of the growth dynamics are worth noting. First,
the particular initial conditions we've chosen have the growth rate of
consumption declining along the transition path; a different choice
could generate a rising growth rate, although declining growth appears
to be more consistent with the value of life in the model (more on this
below).

Second, consider total factor productivity for the idea production
function. With $\lambda=1$, TFP is $\alpha F(z_t) A_t ^\phi$. Because
we've assumed $\phi=1/2$, this production function has the potential to
exhibit positive TFP growth as knowledge spillovers rise over time.
However, a declining danger threshold can offset this. In steady state,
TFP growth for the idea production function is $\phi g_A + \eta
g_z=-0.33\%$. That is, even though a given number of researchers are
generating more and more candidate ideas over time, the number that get
implemented is actually declining because of safety considerations.

Finally, the steady-state growth rate of 1.33\% can be compared to an
alternative path. It is feasible in this economy to let the
technology-induced mortality rate fall to some arbitrarily low level ---
such as 1 death per billion people --- and then to keep it constant at
that rate forever by maintaining a constant technology cutoff $\bar{z}$.
As this constant cutoff gets arbitrarily small, the steady state growth
rate of the economy converges to $\lambda \sigma \bar{n}/(1-\phi)$ ---
that is, to the rule-of-thumb growth rate from
Proposition~\ref{prop:rule}. For our choice of parameter values, this
feasible steady-state growth rate is 4.0\% per year. That is, concerns
for safety make it optimal in this environment to slow growth
considerably relative to what is possible in the steady state.

The reason for this, of course, is the rising value of life, shown for
this example in Figure~\ref{fig:v}.
\begin{figure}[tp]
\caption{The Value of Life: Benchmark Case}
\label{fig:v}
%  \centering \includegraphics[width=3.5in]{cost2c.ps} 
\fignote{4.5in}{Note: Simulation results for the competitive equilibrium
using the parameter values from Table~\ref{tab:params}.  The value of
life rises faster than consumption.}
\end{figure} 
The value of life begins in period 0 at about 100 times annual
consumption; if we think of per capita consumption as \$30,000 per year,
this corresponds to a value of life of \$3 million, very much in the
range considered in the literature
\cnp{ViscusiAldy2003,AshenfelterGreenstone2004,MurphyTopel2005}. Over
time, the value of life relative to consumption rises exponentially at a
rate that converges to 0.67\%, the same rate at which mortality
declines.


\subsection{Numerical Example When $\eta=\infty$}

One element of the model that is especially hard to calibrate is the
distribution from which technological danger is drawn, $F(z)$. The
previous example assumed an exponential distribution so that $\eta=1$;
in particular, the elasticity of the distribution as $z$ approaches zero
is finite. However, this need not be the case. Both the Fr\'{e}chet and
the lognormal distributions feature an infinite elasticity. In
Proposition~\ref{prop:eta}, we showed that this leads the growth rate of
consumption to converge to zero asymptotically. For this example, we
consider the Fr\'{e}chet distribution to illustrate this result:
$F(z)=e^{-z^{-\psi}}$ and we set $\psi=1.1$.\footnote{We require
  $\psi>1$ so that the mean (and hence conditional expectation) exist.
  The elasticity of this cdf is $\eta(z)=\psi z^{-\psi}$, so a small
  value of $\psi$ leads the elasticity to rise to infinity relatively
  slowly.} Other parameter values are unchanged from the benchmark case
shown in Table~\ref{tab:params}, except we now set $\bar{\delta}=1$,
which is needed to put the value of life in the right ballpark.

Figure~\ref{fig:dynFrech} shows the dynamics of the economy for this example.
\begin{figure}[tp]
\caption{Equilibrium Dynamics: Fr\'{e}chet Case}
\label{fig:dynFrech}
%  \centering \includegraphics[width=3.5in]{cost4b.ps}
  \fignote{4.5in}{Note: Dynamics when $F(z)$ is Fr\'{e}chet, so
    $\eta=\infty$: growth slows to zero asymptotically. See notes to
    Figure~\ref{fig:dynamics}.}
\end{figure} 
The growth rate of consumption now converges to zero as $\eta(z)$ gets
larger and larger, meaning that a given decline in the danger threshold
eliminates more and more potential ideas. Interestingly, this rising
elasticity means that the danger threshold itself declines much more
gradually in this example.

Figure~\ref{fig:vFrech} --- with its logarithmic scale --- suggests that
this declining consumption growth rate occurs as consumption gets
arbitrarily high.
\begin{figure}[tp]
\caption{Consumption and the Value of Life: Fr\'{e}chet Case}
\label{fig:vFrech}
%  \centering \includegraphics[width=3.5in]{cost4c.ps} 
\fignote{4.5in}{Note: Dynamics when $F(z)$ is Fr\'{e}chet, so
    $\eta=\infty$: consumption still rises to infinity. See notes to
    Figure~\ref{fig:v}.}
\end{figure} 
The value of life still rises faster than consumption, but the increase
is no longer exponential.

\section{Discussion and Evidence}

The key mechanism at work in this paper is that the marginal utility of
consumption falls quickly, leading the value of life to rise faster than
consumption. This tilts the allocation in the economy away from
consumption growth and toward preserving lives. Exactly this same
mechanism is at work in \cn{HallJones2007}, which studies health
spending. In that paper, $\gamma>1$ leads to an income effect: as the
economy gets richer over time (exogenously), it is optimal to spend an
increasing fraction of income on health care in an effort to reduce
mortality. The same force is at work here in a very different context.
Economic growth combines with sharply diminishing marginal utility to
make the preservation of life a luxury good. The novel finding is that
this force has first-order effects on the determination of economic
growth itself.


\subsection{Empirical Evidence on the Value of Life}

Direct evidence on how the value of life has changed over time is
surprisingly difficult to come by. Most of the empirical work in this
literature is cross-sectional in nature; see \cn{ViscusiAldy2003} and
\cn{AshenfelterGreenstone2004}, for example.  Two studies that do
estimate the value of life over time are \cn{CostaKahn2004} and
\cn{HammittLiuLiu2000}. These studies find that the value of life rises
roughly twice as fast as income, supporting the basic mechanism in this
paper. 

Less direct evidence may be obtained by considering our changing
concerns regarding safety.  It is a common observation that parents
today are much more careful about the safety of their children than
parents a generation ago.  Perhaps that is because the world is a more
dangerous place, but perhaps it is in part our sensitivity to that
danger which has changed.  

I am searching for formal data on how safety standards have changed over
time and how they compare across countries. One source of information
comes from looking at accidental deaths from drowning. Perhaps to a
greater extent than for other sources of mortality, it does not seem
implausible that advances in health technologies may have had a small
effect on drowning mortality: if one is underwater for more than several
minutes, there is not much that can be done. Nevertheless, there have
been large reductions in the mortality rate from accidental drowning in
the United States, as shown in Figure~\ref{fig:drown}.
\begin{figure}[tp]
\caption{Mortality Rate from Accidental Drowning}
\label{fig:drown}
%\centering \includegraphics[width=3.5in]{drowning.ps} 
\fignote{4.5in}{Note: Taken from various issues of the National Center
  for Health Statistics, Vital Statistics Data.  Breaks in the data
  imply different sources and possibly differences in methodology.}
\end{figure} 
At least in part, these are arguably due to safety improvements.

% Tom Philipson (5/31/08) suggests another possible data source:
% spending on the FDA as a share of GDP (chad: or OSHA).  Problem: it's
% very naive to equate this spending with safety?

Safety standards also appear to differ significantly across countries,
in a way that is naturally explained by the model. While more formal
data is clearly desirable, different standards of safety in China and
the United States have been vividly highlighted by recent events in the
news. Eighty-one deaths in the United States have been linked to the
contamination of the drug heparin in Chinese factories
\cnp{MundyWSJ2008}. In the summer of 2007, 1.5 million toys manufactured
for Mattel by a Chinese supplier were recalled because they were
believed to contain lead paint \cnp{SpencerYeWSJ2008}. And in an article
on the tragic health consequences for workers producing toxic cadmium
batteries in China, the {\it Wall Street Journal} reports
\begin{quote}
As the U.S. and other Western nations tightened their regulation of
cadmium, production of nickel-cadmium batteries moved to less-developed
countries, most of it eventually winding up in China. ``Everything was
transferred to China because no one wanted to deal with the waste from
cadmium,'' says Josef Daniel-Ivad, vice president for research and
development at Pure Energy Visions, an Ontario battery company. 
\cnp{CaseyZamiskaWSJ2007}
\end{quote}


\subsection{The Environmental Kuznets Curve}

Another interesting application of the ideas in this paper is to the
environmental Kuznets curve. As documented by \cn{SeldenSong94} and
\cn{GrossmanKrueger95}, pollution exhibits a hump-shaped relationship
with income: it initially gets worse as the economy develops but then
gets better.  To the extent that one of the significant costs of
pollution is higher mortality --- as the Chinese cadmium factory reminds
us --- the declines in pollution at the upper end of the environmental
Kuznets curve are consistent with the mechanism in this paper.  As the
economy gets richer, the value of life rises substantially and the
economy features an increased demand for safety.

In fact, the consequences for economic growth are also potentially
consistent with the environmental Kuznets curve. One of the ways in
which pollution has been mitigated in the United States is through the
development of new, cleaner technologies. Examples include scrubbers
that remove harmful particulates from industrial exhaust and catalytic
converters that reduce automobile emissions. Researchers can spend their
time making existing technologies safer or inventing new technologies.
Rising concerns for safety lead them to divert effort away from new
inventions, which reduces the output of new varieties and slows growth.


\section{Conclusion}

Safety is a luxury good. For a large class of standard preferences used
in applied economics, the value of life rises faster than consumption.
The marginal utility associated with more consumption on a given day
runs into sharp diminishing returns, and adding additional days of life
on which to consume is a natural, welfare-enhancing response. Economic
growth therefore leads to a disproportionate concern for safety.

This force is so strong, in fact, that concerns for safety eventually
outweigh a society's demand for economic growth.  In the economy studied
here, safety considerations lead to a conservative bias in technological
change that slows growth considerably relative to what could otherwise
be achieved.  Depending on exactly how the model is specified, this can
take the form of an overall reduction in exponential growth to a lower
but still positive rate.  Alternatively, the exponential growth rate
itself may be slowed to zero.  

From the standpoint of the growth literature, this is a somewhat
surprising result. Some literatures focus on the importance of finding
policies to increase the long-run growth rate; others emphasize the
invariance of long-run growth to policies.  Hence, the result that
concerns for safety lead to a substantial reduction in the optimal
growth rate is noteworthy.  

The finding can also be viewed from another direction, however. A
literature on sustainability questions the wisdom of economic growth;
for example, see \cn{Ehrlich68}, \cn{Meadowsetal72}, and \cn{Mishan93}.
The model studied here permits very strong concerns for safety and human
life. And while the consequences are slower rates of economic growth ---
even rates that slow to zero asymptotically --- it is worth noting that
the key driving force in the model is an income effect that operates
only as consumption goes to infinity. That is, even the most aggressive
slowing of growth found in this paper features unbounded growth in
individual consumption; it is never the case that all growth should
cease entirely.

This paper suggests a number of different directions for future research
on the economics of safety. It would clearly be desirable to have
precise estimates of the value of life and how this has changed over
time; in particular, does it indeed rise faster than income and
consumption? More empirical work on how safety standards have changed
over time --- and estimates of their impacts on economic growth ---
would also be valuable. Finally, the basic mechanism at work in this
paper over time also applies across countries. Countries at different
levels of income may have very different values of life and therefore
different safety standards. This may have interesting implications for
international trade, standards for pollution and global warming, and
international relations more generally.




\appendix
\section{Appendix: Proofs of the Propositions}
\label{app:proofs}

This appendix contains outlines of the proofs of the propositions
reported in the paper.

%%%%%%%%%%%%%%%%%%%%%%%%%%%%%%%%%%%%%
\hypertarget{proof:rule}{} 
\Proof{prop:rule}{BGP under the Rule of Thumb Allocation} 
%%%%%%%%%%%%%%%%%%%%%%%%%%%%%%%%%%%%%

Equations~\eqref{eq:drule} and \eqref{eq:gNrule} follow immediately from
the setup. The growth rate of ideas then comes from taking logs and
derivatives of both sides of the following equation, evaluated along a
balanced growth path, and using the fact that $\delta^* = \bar{\delta}
g_A (1-\bar{s}) \Gamma(\bar{z})$:
\[
\growth{A} = \alpha F(z_t) \frac{\bar{s}^\lambda N_t^{\lambda}}{A_t ^{1-\phi}}.
\]
\hyperlink{prop:rule}{QED}.

%%%%%%%%%%%%%%%%%%%%%%%%%%%%%%%%%%%%%
\hypertarget{proof:CE}{}
\Proof{prop:CE}{Equilibrium Balanced Growth}
%%%%%%%%%%%%%%%%%%%%%%%%%%%%%%%%%%%%%

Solving the optimization problems that help define the equilibrium and
making some substitutions leads to the following seven key equations
that pin down the equilibrium values of
$\{c_t,s_t,v_t,z_t,A_t,\delta_t,N_t\}$:
\begin{equation}
\label{foc:c}
\clr{ c_t = A_t ^\sigma (1-s_t)  }
\end{equation}
\begin{equation}
\label{foc:s}
\clr{ \frac{s_t}{1-s_t} = \beta - \frac{v_t \delta_t}{c_t} = \beta (1-\frac{\Gamma(z)}{z})  }
\end{equation}
\begin{equation}
\label{foc:v}
\clr{ v_t = \frac{u(c_t)/u'(c_t)}{\rho+\delta_t+\gamma g_{ct} - g_{vt}}  }
\end{equation}
\begin{equation}
\label{foc:z}
\clr{ z_t = \frac{\beta c_t}{v_t \bar\delta g_{At} (1-s_t)}  }
\end{equation}
\begin{equation}
\label{foc:A}
\clr{ \growth{A} = \alpha F(z_t) \frac{s_t ^\lambda N_t ^\lambda}{A_t ^{1-\phi}}  }
\end{equation}
\begin{equation}
\label{foc:dlta}
\clr{ \delta_t = \bar\delta g_{At} (1-s_t) \Gamma(z_t)  }
\end{equation}
\begin{equation}
\label{foc:N}
\clr{ \growth{N} = \bar{n} - \delta_t  }
\end{equation}

We begin by studying the value of life, in equation~\eqref{foc:v}. With
our specification of utility, $u(c_t)/u'(c_t)=\bar{u} c_t ^\gamma +
c_t/(1-\gamma)$. Since $\gamma>1$, the growth rate of the value of life
must be equal to $\gamma g_c$ along an (asymptotic) balanced growth
path.  Equation~\eqref{foc:v} then implies the last result in the
proposition, namely that $u'(c_t) v_t \rightarrow \bar{u}/\rho$.

The fact that $g_v=\gamma g_c$ immediately implies from \eqref{foc:z}
that the danger cutoff $z_t$ converges to zero along a balanced growth
path, because $v_t/c_t$ rises to infinity with $\gamma>1$. Similarly,
$g_z=-(\gamma-1)g_c$. And since $z_t \rightarrow 0$,
equation~\eqref{foc:dlta} implies that $\delta_t \rightarrow 0$ as well.

To get the growth rate of the economy as a whole, recall that
\[
\growth{A} = \alpha F(z_t) \frac{\bar{s}^\lambda N_t^{\lambda}}{A_t ^{1-\phi}}.
\]
Taking logs and derivatives of this equation along the balanced growth
path, and using the fact that $\eta \equiv \lim_{z \rightarrow 0} F'(z)
z/F(z)$ is finite from \hyperlink{ASSUME}{Assumption A}, we have
\[
(1-\phi) g_A = \eta g_z +\lambda \bar{n}.
\]
The growth rate results of the proposition then follow quickly, after we
note that $g_c =\sigma g_A$ and $g_z=-(\gamma-1)g_c$.  For example
\[
 g_A = \frac{\lambda \bar{n}}{1-\phi+\eta \sigma (\gamma-1)}.
\]

Finally, the share of labor devoted to research in steady state can be
found from equation~\eqref{foc:s}. By L'hopital's rule, $\lim
\Gamma(z)/z = \Gamma'(0)$. Using the definition of the conditional
expectation, one can calculate that
\[  \Gamma'(z) = \left(1-\frac{\Gamma(z)}{z} \right) \eta(z) \]
where $\eta(z) \equiv zF'(z)/F(z)$. Taking the limit as $ z\rightarrow
0$ and noting that $\eta$ is finite reveals that $\lim
\Gamma(z)/z=\eta/(1+\eta)$.  Substituting this into~\eqref{foc:s} yields
the asmptotic value for $s$. \hyperlink{prop:CE}{QED}.



%%%%%%%%%%%%%%%%%%%%%%%%%%%%%%%%%%%%%
\hypertarget{proof:eta}{}
\Proof{prop:eta}{Equilibrium Growth with $\eta=\infty$} 
%%%%%%%%%%%%%%%%%%%%%%%%%%%%%%%%%%%%%

First, we show $c_t \rightarrow \infty$, by contradiction. Suppose not.
That is, suppose $c_t \rightarrow c^* \in (0,\infty)$. The contradiction
arises because the model has a strong force for idea growth and
therefore consumption growth. By~\eqref{foc:v}, $v_t \rightarrow v^*$
and \eqref{foc:s} and \eqref{foc:z} imply that $s_t \rightarrow s^* \in
(0,1)$ and $z_t \rightarrow z^*>0$. Studying the system of differential
equations in \eqref{foc:A}, \eqref{foc:dlta}, and \eqref{foc:N} reveals
that
\[
\growth{A} \rightarrow \frac{\lambda \bar{n}}{1-\phi + \bar{\delta} (1-s^*) \Gamma(z^*)} >0.
\]
But since $c_t=A_t ^\sigma (1-s_t)$, this means $c_t \rightarrow
\infty$, which contradicts our original supposition that $c_t
\rightarrow c^*$. Therefore this supposition was wrong, and $c_t$ does
in fact go to $\infty$.

Next, we show everything else, such as $g_{ct} \rightarrow 0$.  With
$c_t \rightarrow \infty$ and $\gamma>1$, $v_t/c_t \rightarrow \infty$
from \eqref{foc:v}, as the value of life rises faster than consumption.
Then \eqref{foc:z} implies that
\[
\frac{v_t}{c_t} = \frac{\beta}{\bar\delta z_t g_{At} (1-s_t)}
 = \frac{\beta (1+\beta(1-\Gamma(z_t)/z_t)}{\bar\delta z_t g_{At}}
\]
But then $v_t/c_t \rightarrow \infty$ if and only if $z_t g_{At}
\rightarrow 0$.

First, we show that $z_t$ has to go to zero. Why? Suppose not. That is,
suppose $z_t \rightarrow z^*$ and $g_{At} \rightarrow 0$. From
\eqref{foc:s}, $s_t \rightarrow s^* \in (0,1)$. And then from
\eqref{foc:dlta}, it must be that $\delta \rightarrow 0$. But then
population grows at rate $\bar{n}$ eventually and the law of motion for
ideas \eqref{foc:A} would lead to exponential growth in $A_t$, which is
a contradiction. Therefore $z_t$ has to go to zero.

Is it possible that $g_{At}$ does not then also go to zero? No. Notice
that $z_t \rightarrow 0$ as rapidly as consumption (and therefore $A_t$)
go to infinity. But the fact that $\eta = \infty$ means that $F(z_t)$
goes to zero faster than $N_t$ and $A_t$ are rising.

The fact that $s_t \rightarrow 0$ comes from \eqref{foc:s} because, as
we show next, $\lim_{z \rightarrow 0} \Gamma(z)/z = 1$ when
$\eta=\infty$.  By L'hopital's rule, $\lim \Gamma(z)/z = \Gamma'(0)$.
Using the definition of the conditional expectation, one can calculate
that 
\[  \Gamma'(z) = \left(1-\frac{\Gamma(z)}{z} \right) \eta(z) \]
where $\eta(z) \equiv zF'(z)/F(z)$.  Since $\eta(z) \neq 0$, we can
divide both sides by $\eta(z)$ and consider the limit as $z \rightarrow
0$:
\[ \lim \frac{\Gamma'(z)}{\eta(z)} = 1 - \lim \frac{\Gamma(z)}{z}. \]
The left-hand size is zero since $\eta(z) \rightarrow \infty$ by
assumption, which proves the result.  

The fact that $z_t \rightarrow 0$, $g_{At} \rightarrow 0$, and $s_t
\rightarrow 0$ imply that $\delta_t \rightarrow 0$ and $g_{ct}
\rightarrow 0$.
\hyperlink{prop:eta}{QED}.


%%%%%%%%%%%%%%%%%%%%%%%%%%%%%%%%%%%%
\hypertarget{proof:gamma}{}
\Proof{prop:gamma}{Equilibrium Growth with $\gamma<1$}
%%%%%%%%%%%%%%%%%%%%%%%%%%%%%%%%%%%%

The equilibrium with $\gamma<1$ is characterized by the same seven
equations listed above in the proof of Proposition~\ref{prop:CE},
equations~\eqref{foc:c} through~\eqref{foc:N}. The proof begins in the
same way, by studying the value of life in equation~\eqref{foc:v}. With
our specification of utility, $u(c_t)/u'(c_t)=\bar{u} c_t ^\gamma +
c_t/(1-\gamma)$. Since $\gamma<1$, the constant term disappears
asymptotically and the growth rate of the value of life equals $g_c$
along an (asymptotic) balanced growth path. Equation~\eqref{foc:v} then
implies the last result in the proposition, giving the constant ratio of
the value of life to consumption.

The fact that $v_t/c_t$ converges to a constant means that $z_t$
converges to a nonzero value, according to equation~\eqref{foc:z}.
Similarly, $\delta_t$ does as well, according to
equation~\eqref{foc:dlta}. The solution for the growth rate and the
research share are found in ways similar to those in the
\hyperlink{proof:CE}{proof} of Proposition~\ref{prop:CE}.
\hyperlink{prop:gamma}{QED}.


%%%%%%%%%%%%%%%%%%%%%%%%%%%%%%%%%%%%
\hypertarget{proof:swf}{}
\Proof{prop:swf}{Optimal Balanced Growth}
%%%%%%%%%%%%%%%%%%%%%%%%%%%%%%%%%%%%

The Hamiltonian for the optimal growth problem is
\[
H=N_t u(c_t) + \mu_{1t} \alpha F(z_t) s_t ^\lambda N_t ^\lambda A_t^\phi 
 + \mu_{2t} N_t \left( \bar{n}-\bar{\delta} \alpha F(z_t) s_t ^\lambda N_t ^\lambda A_t^{\phi-1} 
 (1-s_t) \int_0 ^{z_t} zf(z)dz  \right).
\]
Applying the Maximum Principle, the first order necessary conditions are\\
\begin{tabular*}{\textwidth}{cc}
\clr{$H_s=0$:} & $N_t u'(c_t) A_t ^\sigma = \mu_{1t} \lambda
\frac{\dot{A}_t}{s_t} - \mu_{2t} N_t ( \lambda \frac{\delta_t}{s_t} - \frac{\delta_t}{1-s_t})$\\
\clr{$H_z=0$:} & $\mu_{1t}=\mu_{2t} N_t \bar\delta z_t \cdot \frac{1-s_t}{A_t}$\\
\clr{Arbitrage($A_t$):} & $\rho=\frac{\dot\mu_{1t}}{\mu_{1t}} + \frac{1}{\mu_{1t}}
\left( N_t u'(c_t) \sigma \frac{c_t}{A_t} + \mu_{1t} \phi \growth{A} -
\mu_{2t} N_t (\phi-1) \frac{\delta_t}{A_t} \right)$\\
\clr{Arbitrage($N_t$):} & $\rho=\frac{\dot\mu_{2t}}{\mu_{2t}} +
\frac{1}{\mu_{2t}}
\left( u(c_t)+\mu_{1t} \lambda
\frac{\dot{A}_t}{N_t}+\mu_{2t}(\bar{n}-\delta_t)-\mu_{2t} N_t \lambda \frac{\delta_t}{N_t} \right)$
\end{tabular*}
together with two transversality conditions: $\lim_{t \rightarrow
  \infty} \mu_{1t} A_t e^{-\rho t}=0$ and $\lim_{t \rightarrow
  \infty} \mu_{2t} N_t e^{-\rho t}=0$.

Combining these first order conditions (use the first and second to get
an expression for $\mu_{2t}$ and substitute this into the arbitrage
equation for $N_t$) and rearranging yields:
\begin{equation}
\frac{ \rho-g_{\mu2t}-(\bar{n}-\delta_t) + \lambda \delta_t - \lambda g_{At}
  \bar\delta (1-s_t) z_t  }  %numerator
{ \lambda \bar\delta g_{At} z_t \frac{1-s_t}{s_t} -
  \frac{\delta_t}{1-s_t}
  (\lambda \frac{1-s_t}{s_t}-1)  }  % denominator
= \frac{u(c_t)}{u'(c_t)c_t} \cdot (1-s_t).
\end{equation}
This is the key equation for determining the asymptotic behavior of
$z_t$. In particular, along a balanced growth path, the right-hand-side
goes to infinity for $\gamma>1$. This requires that $z_t \rightarrow 0$
so that the denominator of the left side goes to zero (since $\delta_t
\rightarrow 0$ as well).  Moreover, with some effort, one can show that
the denominator on the left side grows at the same rate as $z_t$ along
the balanced growth path, which implies that $g_z=-(\gamma-1) g_c$ from
the usual value-of-life argument used earlier.

The result for the growth rate of $A_t$ and $c_t$ follows by the same
arguments as in the proof of Proposition~\ref{prop:CE}.  Finally, one
can combine the first order conditions to solve for the allocation of
research. \hyperlink{prop:swf}{QED}.

\section{Appendix: Solving the Model Numerically}
\label{app:solve}

The transition dynamics of the equilibrium allocation can be studied as
a system of four differential equations in four ``state-like'' variables
that converge to constant values: $\ell_t$, $m_t$, $\delta_t$, and
$w_t$. These variables, their meaning, and their steady-state values are
displayed in Table~\ref{tab:solve}.
\begin{table}[tp]
\caption{Key ``State-Like'' Variables for Studying Transition Dynamics}
\label{tab:solve}
\begin{tabular*}{\textwidth}{@{\extracolsep{\fill}}cp{2in}c} \hline
 Variable & Meaning & Steady-State Value  \\ \hline
$\ell_t \equiv \frac{v_t \delta_t}{c_t}$ & Value of life $\times$ mortality 
  & $\ell^* = \beta \cdot \frac{\eta}{1+\eta}$ \\
$m_t \equiv g_{At}$ & Growth rate of $A_t$ & $m^*=g_A$ \\
$\delta_t$ & Mortality rate & $\delta^*=0$ \\
$w_t \equiv \frac{u(c_t)}{u'(c_t) c_t} \cdot \frac{c_t}{v_t}$
  & Value of a year of life relative to mortality price
  & $w^*=\rho$ \\
\hline
\end{tabular*}
\end{table}

Letting a ``hat'' denote a growth rate, the laws of motion for these
state-like variables are
\begin{equation}
\hat{\ell}_t = \frac{\rho+\delta_t-w_t+(\gamma-1)\sigma
  m_t+\lambda(\bar{n}-\delta_t)-(1-\phi)m_t }  % numerator
 {1+k_t
   \left(\frac{\lambda}{\beta-\ell_t}-\gamma \right)  
   +\frac{\eta(z_t)+\theta(z_t)}{1-\theta(z_t)} } % denominator
\end{equation}
\begin{equation}
\hat{m}_t = -\left(\frac{\eta(z_t)}{1-\theta(z_t)}+
  \frac{\lambda k_t}{\beta-\ell_t} \right) \hat{\ell}_t +
  \lambda(\bar{n}-\delta_t)-(1-\phi)m_t
\end{equation}
\begin{equation}
\hat{\delta}_t=\hat{m}_t+ \left( k_t-\frac{\theta(z_t)}{1-\theta(z_t)}
      \right) \hat{\ell}_t
\end{equation}
\begin{equation}
\hat{w}_t = \hat{\delta}_t-\hat{\ell}_t+\left(\gamma-1+\frac{\delta_t}{w_t
  \ell_t} \right) \left( \sigma m_t+k_t \hat{\ell}_t  \right)
\end{equation}
where $\theta(z) \equiv z\Gamma'(z)/\Gamma(z)=\eta(z) (z/\Gamma(z)-1)$
is the elasticity of the conditional expectation function and $k_t
\equiv \ell_t/(1+\beta-\ell_t)$. The only other variable that must be
obtained in order to solve these differential equations is $z_t$, and it
can be gotten as follows. First, $s_t =
(\beta-\ell_t)/(1+\beta-\ell_t)$. With $s_t$ in hand, $\Gamma(z_t)$ can
be recovered from the state-like variables using the mortality rate:
$\delta_t=\bar{\delta} m_t (1-s_t) \Gamma(z_t)$. Finally, $z_t =\beta
\Gamma(z_t)/ \ell_t$.

We solve the system of differential equations using ``reverse
shooting''; see \citet[p. 355]{Judd98}.  That is, we start from the steady
state, consider a small departure, and then run time backwards.  For the
results using the exponential distribution, we set $T=600$; for the
results using the Fr\'{e}chet distribution, we set $T=12150$.

An interesting feature of the numerical results is that $\hat{\ell}_t
\approx 0$ holds even far away from the steady state. The reason is that
$\lim_{z \rightarrow 0} \theta(z)=1$: if $z$ changes by a small percent
starting close to zero, the conditional expectation changes by this same
percentage. But this means that $\hat{\ell}_t \approx 0$ since there is
a $1/(1-\theta(z_t))$ term in the denominator. But since
$z_t/\Gamma(z_t)=\beta/\ell_t$, if $\ell_t$ does not change by much,
then $z_t$ will not change by much either. \hyperlink{APPB}{QED}.

{\small
\bibliography{Growth.bib}
}

\end{document}
