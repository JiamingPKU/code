\section{美国FOF市场总资产}
\subsection{ARIMA建模}
首先使用ADF检验, 在备择假设为平稳性的条件下, 对FOF基金的资产总量数据进行检验. 检验结果为$P=0.8158$, 这说明FOF的资产总量数据并不是一个平稳的时间序列. 而对FOF资产总量取对数差分后,即得到总资产的对数增长率序列${GR\_ast_t}$, 再次进行ADF检验, 检验结果$P<0.01$, 拒绝了非平稳的原假设, 即其对数差分后是一个平稳序列.

\textcolor{red}{展示: AST和GR\_ast的ADF检验结果, by王喆. }

下面对对数差分后的序列进行ARMA建模. 绘制$GR\_ast_t$的自相关和偏自相关图像可以发现, 此序列的ACF函数在5阶处截尾, PACF函数在5阶处结尾. \emph{经过反复尝试,} 当使用MA(5)对序列进行刻画时, 可以得到较好的估计效果. MA(5)模型的极大似然估计结果如下:

\textcolor{red}{展示: 定阶所用的数据或其图示, by王喆}

\textcolor{red}{展示: MA(5)的极大似然估计结果的表格或图示, by王喆}

\subsection{模型诊断}

对估计的残差$\hat{u}_t$进行Ljung-Box检验, $p=0.84$, 可以接受原假设, 满足白噪声要求. 同时绘制$\hat{u}_t$的自相关函数, 从1阶开始都不显著, 也说明$\hat{u}_t$序列不存在自相关.

\textcolor{red}{展示: 残差的白噪声检验结果及其ACF、PACF图示, by王喆}

继续对$\hat{u}_t^2$进行 McLeod.Li检验, 判断是否存在ARCH效应. 检验结果各阶的$P$值都接近1, 说明不存在ARCH效应.

\textcolor{red}{残差平方的第一次检验结果或其图示, by王喆}

但是, 如果绘制出标准化的残差图进行观察, 会发现在第90期有一个明显的异常值. 很有可能因为这个异常值的出现, 使得其他的波动被隐藏, 在模型诊断的检验中造成了偏差. 通过Bonferroni法则进行检验MA(5)模型, 在第90期存在一个强影响点$GR\_ast_{90}$. 这进一步确认了我们的猜测.

\textcolor{red}{在上一段补充: Bonferroni法则的简要介绍, 用几句话说一下就好, by王喆}

\textcolor{red}{展示: 异常值的图示\&各种诊断情况, by王喆}

\subsection{异常值处理}
为了削弱第90期的异常值对模型的影响, 令
$$GR\_ast_{90} = \frac{1}{3} \cdot (GR\_ast_{89}+GR\_ast_{90}+GR\_ast_{91})$$
重新对${GR\_ast_t}$序列进行建模估计.
此时对使用极大似然估计得到的残差序列${\hat{u}_t^2}$进行 McLeod.Li检验, 检验结果$P$值很小, 拒绝了不存在条件异方差的原假设, 即存在GARCH效应. 于是使用 ARMA(0,5)-GARCH(0,1) 对调整后的${GR\_ast_{t}}$进行建模. 估计结果如下:

\textcolor{red}{写出模型的完整表达式,展示每个参数的值和标准差,by王喆}

\textcolor{red}{展示: 最新模型的预测结果,就是beamer里面的那一堆图,by王喆}

\textcolor{red}{修改: 统一一下上面几段用到的ARCH和GARCH字样,by王喆}

